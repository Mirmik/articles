\documentclass[a4paper]{article}

\usepackage[12pt]{extsizes}
\usepackage[left=30mm, top=25mm, right=20mm, bottom=25mm, nohead]{geometry}

\usepackage{xcolor}

\usepackage{amssymb}
\usepackage{amsmath}
\usepackage{lipsum}
\setlength{\parindent}{3ex}
\setlength{\parskip}{1em}

\usepackage[utf8]{inputenc}
\usepackage[T2A]{fontenc}
\usepackage[english, russian]{babel}
\usepackage{caption}

\usepackage{graphicx} 
\graphicspath{ {./images/} }

\definecolor{shadecolor}{RGB}{230,230,230}

\begin{document}

% НАЧАЛО ТИТУЛЬНОГО ЛИСТА
\begin{center}
\hfill \break
\hfill\break
\hfill \break
\hfill \break
\hfill \break
\hfill \break
\hfill \break
\hfill \break
\hfill \break
\hfill \break
\large{<<Компенсация силовых невязок в тензорных системах управления.>>}\\
\hfill \break
\hfill \break
\hfill \break
\hfill \break
\hfill \break
\hfill \break
\hfill \break
\hfill \break
\end{center}
 
\normalsize{ 
\begin{tabular}{lccl}
Автор: Н.Ф. Сорокин\\\\
\end{tabular}
}\\
\hfill \break
\hfill \break
\hfill \break
\hfill \break
\hfill \break
\begin{center} \today \end{center}
\thispagestyle{empty} % выключаем отображение номера для этой страницы
\newpage
% КОНЕЦ ТИТУЛЬНОГО ЛИСТА

\section{Постановка задачи.}
Пусть даны два или более многозвенных манипуляторов, удерживающих груз в некоем неподвижном положении или управляющих перемещением данного груза в пространстве. 

С точки зрения количества степеней свободы система является статически неопределимой.

Задача ставится в построении такой системы управления, которая обеспечит надлежащее управление положением груза.

\section{Исходная система управления.}
В качестве исходной системы управления возьмём систему управления положением концевого звена манипулятора в следующем виде:

Данная система способна обеспечить захват и управление положением захваченного груза, но в условиях статической неопределимости может иметь нежелательные побочные эффекты в виде силовой невязки и даже потери устойчивости.

\section{Статический анализ исходной системы.}


\section{Анализ динамики исходной системы.}
После того как второй манипулятор захватывает груз, совокупная система управления по силам принимает вид.

В данной системе мы можем обнаружить перекрёстную положительную обратную связь.

\section{Компенсация силовой невязки.}
Для компенсации силовой невязки необходимо ввести в систему управления сигнал, работающий в противоположном невязке направлении. 

К сожалению, проводить анализ геометрии задачи в общем случае не представляется возможным, а потому предложим более простой и универсальный метод. Вместо сигнала уменьшающего невязку введём сигнал противоположный полному вектору реакции, тоесть сумме невязки и полезного воздействия. Это будет своего рода расслабляющий сигнал, то есть сигналвсегда стремящийся заставить систему уйти от нагрузки. Анализ и моделирование показывают, что благодаря обратной связи по положению, несмотря на наличие расслабляющего сигнала, целевое управление все равно будет достигнуто, правда, возникает некоторая статическая ошибка пропорциональная коэффициенту расслабления.

Статическая ошибка может быть выбрана путём внесения интегральной состовляющей в регулятор. Строго говоря, скорее всего, этого не потребуется, поскольку интегральные составляющие уже есть в системах управления приводов.

\section{Выводы.}
Исходя из проведенного исследования, учет силовых невязок в системах тензорного управления манипуляторами легко разрешается универсальным методом введения добавки управляющего сигнала, соноправленного мотору реакции в захвате.

Большим плюсом данного способа является его независимость от геометрии задачи и количества манипуляторов в системе.

\end{document}

