\documentclass[a4paper]{article}

\usepackage[12pt]{extsizes}
\usepackage[left=30mm, top=25mm, right=20mm, bottom=25mm, nohead]{geometry}

\usepackage{xcolor}

\usepackage{amssymb}
\usepackage{amsmath}
\usepackage{lipsum}
\setlength{\parindent}{3ex}
\setlength{\parskip}{1em}

\usepackage[utf8]{inputenc}
\usepackage[T2A]{fontenc}
\usepackage[english, russian]{babel}
\usepackage{caption}

\usepackage{graphicx} 
\graphicspath{ {./images/} }

\definecolor{shadecolor}{RGB}{230,230,230}

\begin{document}

% НАЧАЛО ТИТУЛЬНОГО ЛИСТА
\begin{center}
\hfill \break
\hfill\break
\hfill \break
\hfill \break
\hfill \break
\hfill \break
\hfill \break
\hfill \break
\hfill \break
\hfill \break
\large{<<Тензорные сигналы в системах автоматического управления.>>}\\
\hfill \break
\hfill \break
\hfill \break
\hfill \break
\hfill \break
\hfill \break
\hfill \break
\hfill \break
\end{center}
 
\normalsize{ 
\begin{tabular}{lccl}
Автор: Н.Ф. Сорокин\\\\
\end{tabular}
}\\
\hfill \break
\hfill \break
\hfill \break
\hfill \break
\hfill \break
\begin{center} \today \end{center}
\thispagestyle{empty} % выключаем отображение номера для этой страницы
\newpage
% КОНЕЦ ТИТУЛЬНОГО ЛИСТА

\section{Введение. Преимущества тензорных сигналов.}

Известно, что системы управления работающие в условиях многомерного пространства зачастую имеют большое количество перекрёстных связей между каналами управления. Это объясняется тем, органы управления таких систем обычно связаны с самим объектом управления и эффект оказываемый ими на параметры объекта управления зависит от его текущего состояния.

Вместе с тем скалярные каналы управления не являются естественными для объекта управления, поскольку построенная в терминах скалярных каналов система сильно зависит от выбора этих самых каналов и может работать только в ограниченном множестве линеаризованных режимов.

Для разрешения этой проблемы имеет смысл перейти к рассмотрению построения систем уравления в терминах объектов имеющих завершенный физико-геометрический смысл, то есть в терминах тензоров. Это не в коей мере не означает, что из уравнений стабилизации куда-либо денуться скалярные компоненты, но все они будут подчинены соответствующим тензорам, а следовательно выбор конкретной системы координат для расчета не будет влиять на динамику системы в целом. 

Преимущество перехода к тензорному исчислению состоит в том, что многие перекрёстные связи, оказываются на самом деле всего-лишь следствием линейных преобразований над векторными/тензорными величинами. При переходе к тензорному исчислению такие перекрёстные связи естественным образом исключаются из системы уравнений и оказываются частью операций более высокого уровня, а именно операций над тензорами. Поскольку, в рамках анализа систем управления, работающих с тензорными сигналами, вопрос устойчивости также должен решаться на уровне тензоров, такие перекрёстные связями будут учтены в общем порядке. Тензорный вид уравнений движения и системы стабилизации снижает объёмность выкладок и является естественным для реализации в виде програмного кода. 

Отдельно стоит отметить, что использование векторных моторов и прочих параметров, объединяющих линейные и угловые параметры общей физической природы также уменьшает количество перекрестных связей в системе, поскольку эти компоненты часто оказываются взаимозависимыми при переходах между системами координат, а потому имеет смысл в качестве тензорных сигналов использовать именно такие, объединенные пары угловых и линейных параметров. 

\section{Требования к объектам управления.}
Забегая немного вперёд, можно сказать, что формализм тензорных сигналов удобен для построения систем автоматического управления объектами совершающими сложные движения в условиях изотропного пространства, мерность которого выше единицы.

К таким объектам относятся роботы-манипуляторы, дроны, некоторые виды автомашин, шагающие роботы. 
Данная статья посвящена математическому формализму обобщающему задачи управления этими и другими групами ОУ.

Хочеться отметить интересное свойство СУ, работающих с тензорными сигналами. Некоторые задачи управления, которые мы можем сформулировать в трехмерном пространстве имеют прямые аналоги в пространстве двумерном. В терминах тензоров системы управления, решающие эти две задачи будут очень похожи друг на друга, что выражается в эквивалентности структурных схем СУ. Тоесть, фактически, оказывается возможным построение системы управления независимой от мерности пространства задачи. Поскольку практически важные задачи ограничены двумерным и трёхмерным пространствами (а также одномерным в вырожденном случае), врядли это наблюдение может иметь значимые практические следствия, но является интересным с теоретической точки зрения. 

\section{Обработка тензорных сигналов. Общие положения и принципы.}
Тензорным сигналом будем называть изменяющуюся во времени геометрическую сущность, независимую от выбора системы координат, и однозначно представленную набором (или эквивалентными наборами) своих компонент. 

Особенностью тензорного сигнала является множественность его возможных представлений вплоть до того, что в различных частях системы управления один и тот же тензорный сигнал может быть представлен разным набором компонент (так например, тензор угловой ориентации может быть представлен матрицей поворота, кватернионом или вектором наименьшего поворота и при этом эти формы могут быть заданы в различных базисах). Выбор компонентного представления в вычислительной системе не влияет на динамические свойства системы, поскольку операции над тензорами эквивалентны во всех системах координат.

\section{Тензор ориентации, тензор положения и их производные в цепи обратной связи. }

Тензор ориентации (или тензор поворота) является наиболее общим способом описания углового положения объекта или системы координат. Тензор поворота является тензором второго ранга и представлен матрицей поворота. Эквивалентной формой представления тензора поворота является вектор конечного поворота. Связь между вектором конечного поворота и матрицей поворота определяется следующим образом:

()

Тензор положения является тензором второго ранга и представляет собой тензор поворота расширенный тензором(вектором) трансляции.

()

Пусть U - тензор уставки положения, а X - тензор текущего положения.

Для тензоров положения определены операции сложения/вычитания и умножения(композиции). 
Введем два типа невязок.

Адитивная невязка:
D = U -X

Мультпликативная невязка:
U = XE; 
E = X^{-1}U

Подставив () в () получим:
D = XE - X
D = X(E - I),
где I - единичная матрица.

Матрица в условиях малых невязок E - I имеет вид:
()

D = X | \rho^x 0 \\ r 0\\ | = | X_{rot} \rho^x 0 \\ X_{rot} r 0 |

Угловая компонента представляет из себя антисимметричный тензор и может быть эквивалентно представлена сопряженным вектором, который является вектором поворота мультипликативной невязки:

Таким образом в векторной форме тензор D приближённо имеет вид:
D = X_a (\rho_e, r_e)

Продифференцируем аддитивную невязку:
\dot{D} = \dot{U} -\dot{X}

В работе [] показано, что производная тензора положения может быть представлена в следующей форме
\dot{P} = P S, где S - правый тензор спина.

Тогда 
\dot{D_a} = U_a S_u - X_a S_x
\dot{D_a} = X_aE_a S_u - X_a S_x
\dot{D_a} = X_a(E_a S_u - S_x)

Приняв во внимание, что антисимметричный тензор правого спина может быть эквивалентно представлен вектором правой угловой скорости (), а также воспользовавшись приёмом аналогичным () для векторного вида тензора производной аддитивной невязки имеем:

\dot{D} = X_a (E_a \Omega_u - \Omega_x, V_u - V_x)

Альтернативное разложение () может быть проведено согласно правилу дифферинцирования тензорного произведения:
\dot{D} = \dot{(XE)} -\dot{X} = \dot{X}E + X\dot{E} - \dot{X}
\dot{D} = X(X^{-1}\dot{X}E + \dot{E} - X^{-1}\dot{X})

X^{-1} \dot{X} = 
| X_a^T,  - X_a^T X_l, 
	  0,            1 | \dot{X} = 

| X_a^T \dot{X_a},  - X_a^T X_l \dot{X_l}, 
	  0,            1 |



Из () и () имеем:
\dot{E} = E_a S_u - S_x E_a

Выражение () фактически является формулой Бура,  





????????????????????????????????????????????????
\section{Метод структурных схем в применении к тензорным сигналам.}
Обладая операциями линеаризованного дифференцирования и линеаризованного интегрирования тензорных сигналов, а также операциями сумирования, вычитания и умножения (на скаляр и матрицу), а также операцией композиции, мы можем применить классический метод структурных схем к тензорным сигналам.

Структурные схемы, описывающие сигналы в тензорной форме имеют свои особенности. В частности многие коэффициенты на них являются тензорными, а сигнал ошибки углового положения (контроль положения типичен для систем интересующего нас класса) вычисляется не с помощью разности а согласно процедуры обратной композиции.

Пусть U - линейный оператор уставки углового положения. X - линейный оператор текущего углового положения, а E - линейный оператор относительной ошибки. Тогда

\begin{equation}U = XE\end{equation}

Отсюда
\begin{equation}E = X^{-1}U\end{equation}

Несмотря на то, что обратная композиция не является сложением, при линеаризации в любом из возможных режимов обратная композиция в первом приближении в области малых отклонений будет вести себя как разность $e = u - x$. 

В общем случае, до тех пор, пока мы рассматриваем области малых отклонений, возможно проанализировать систему как многомерную линейную системой управления, что позволяет рассматривать вопросы устойчивости и прочие связанные проблемы известными методами. 

Следует заметить, что, если все операции тензорной СУ удалось свести к действиям над линейными операторами, то по её структурной схеме можно восстановить структурную схему любого частного опорного режима.

Подробное рассмотрение вопроса устойчивости СУ с тензорными сигналами в общем виде выходит за рамки настоящей статьи.
?????????????????????????????????????????????????????

\section{Группы органов управления}
До этого момента мы рассматривали абстроктное тензорное управляющее воздействие, однако в реальности практически не встречаются органы управления способные к генерации многомерного тензорного управляющего воздействия.

Невозможность создать такое управляющее воздействие одним органом управления очевидно разрешается при использовании нескольких совместно действующих органов управления.

Группой органов управления будем называть совокупность органов управления, совместно решающих задачу построения тензора управления U в виде $U = U(v_0, v_1, ..., v_n)$, где $n$ - количество органов управления. Если относительно управляющих воздействий органов управления выполняется принцип суперпозиции, то U является линейной комбинацией. 

\begin{equation}U = a_0v_0 + a_1v_1 + ... + a_iv_i.\end{equation}
\begin{equation}u_i=a^i_jv^j\end{equation}
\begin{equation}U = AV\end{equation}

Уравнение () есть система линейных уравнений к решению которой сводится задача поиска управляюющего воздействия отдельных органов группы. Система () может иметь одно решение, не иметь решений вовсе или же иметь множество решений.

Случай отсутствия решений означает, что желаемое управление, требуемое от группы не может быть выполненно (вероятно, в силу физической несовместимости). 

Случай множества решений означает, что желаемое управление может быть достигнуто множеством способов. 
Поиск одного из множества решений возможен с использованием метода псевдообратной матрицы, однако вероятно, разработчик СУ захочит задать правила выбора конкретного решения из доступного множества.

Поиск оптимального решения на данном множестве требует введения функционала оптимизации и, возможно, дополнительных условий.

\begin{equation}F(V) -> min\end{equation}
\begin{equation}AV = U\end{equation}
\begin{equation}CV <= d\end{equation}

Если F(V) - квадратичный функционал, а дополнительные условия отсутствуют, задача () является задачей квадратичного программирования и может быть разрешена в форме:

\begin{equation}
\begin{vmatrix}
Q & A^T\\
A & 0
\end{vmatrix}
\begin{bmatrix}
x\\
\lambda
\end{bmatrix}
=
\begin{bmatrix}
-c\\
U
\end{bmatrix}
\end{equation}

\section{Матрица чувствительности группы органов управления.}
Вернемся к вопросу поиска матрица A из уравнения ().

Как было сказано выше, уравнение в форме () может быть записано, если относительно воздействия органов управления выполняется принцип суперпозиции. 

Матрица A есть матрица частных производных $\frac{\partial{u^i}}{\partial{v^j}}$.

Практически значимыми примерами групп органов управления являются системы с суперпозицией силовых и мгновенных кинематических воздействий. Для них матрица А формируется из компонент линейного оператора переноса соответствующего воздействия:

Силовой перенос.
Кинематический перенос.

Из формы операторов переноса следует, что линейные и угловые параметры при построении желаемого управления должны рассматриваться совместно. Такой подход свойственнен для винтового исчисления / исчисления векторных моторов. Следует отметить, что векторный мотор является тензором и к нему применимо всё, описанное в предыдущих разделах.  

\section{Учёт неидеальности передаточных функций органов управления.}
В следствие неидеальности органов управления, в структурной схеме между вычислителем и эффектом управления возникает обусловленная динамикой органов управления передаточная функция. Эта функция является диагонально матричной, поскольку каждый орган управления может иметь отличающиеся передаточную функции и постоянную времени. С целью улучшения качества управления следует стремиться выравнить передаточные функции органов управления. Если различия постоянных времени и передаточных функций обусловленно конструкционно, достигнуть этого эффекта можно с помощью введения дополнительной фильтрации управляющих сигналов по каждому отдельному каналу.

\section{Эффект миграции управляющего сигнала.}
Инерционность органов управления также приводит к эффекту миграции управляющего сигнала, выраженного в том, что в случае изменения оператора переноса управляющего воздействия (вследствии каких-либо эволюций системы, например её поворота или изменения плеча), управление накопленное органом управления неуспевает подстроиться под новую матрицу чувствительности и тензор управления получает неучтенную добавку.

Вычислим ее (учитывая, что R - линейный оператор):

\begin{equation}U_vi = R^\times V_i\end{equation}
\begin{equation}\dot{U}_vi = R^\times \frac{\partial{V_i}}{\partial{t}} + \dot{R}^\times V_i\end{equation}

Второе слагаемое в этом уравнении отвечает за неучтенный сигнал миграции управляющего воздействия.  

\section{Вывод.}
Проведенный анализ показывает, что принципиальных проблем для использования тензорных сигналов в системах автоматического управления нет. 

Специфика тензорного управления накладывает некоторые коррективы на семантику математических операций, не нарушая при этом принципов построения систем с обратной связью.

Введение групп органов управления позволяет получить тензорный сигнал управления и замкнуть им систему с тензорной обратной связью. 

Приведенные выкладки не опираются на физические особенности какой-либо конкретной группы объектов управления, а потому могут быть применены к широкому классу систем.

\end{document}

