\documentclass[a4paper]{article}

\usepackage[12pt]{extsizes}
\usepackage[left=30mm, top=25mm, right=20mm, bottom=25mm, nohead]{geometry}

\usepackage{xcolor}

\usepackage{amssymb}
\usepackage{amsmath}
\usepackage{lipsum}
\setlength{\parindent}{3ex}
\setlength{\parskip}{1em}

\usepackage[utf8]{inputenc}
\usepackage[T2A]{fontenc}
\usepackage[english, russian]{babel}
\usepackage{caption}

\usepackage{graphicx} 
\graphicspath{ {./images/} }

\definecolor{shadecolor}{RGB}{230,230,230}

\begin{document}

% НАЧАЛО ТИТУЛЬНОГО ЛИСТА
\begin{center}
\hfill \break
\hfill\break
\hfill \break
\hfill \break
\hfill \break
\hfill \break
\hfill \break
\hfill \break
\hfill \break
\hfill \break
\large{<<Тензорная форма сигналов в системах автоматического управления.>>}\\
\hfill \break
\hfill \break
\hfill \break
\hfill \break
\hfill \break
\hfill \break
\hfill \break
\hfill \break
\end{center}
 
\normalsize{ 
\begin{tabular}{lccl}
Автор: Н.Ф. Сорокин\\\\
\end{tabular}
}\\
\hfill \break
\hfill \break
\hfill \break
\hfill \break
\hfill \break
\begin{center} \today \end{center}
\thispagestyle{empty} % выключаем отображение номера для этой страницы
\newpage
% КОНЕЦ ТИТУЛЬНОГО ЛИСТА

\section{}

Известно, что системы управления работающих в условиях многомерного пространства зачастую имеют большое количество перекрёстных связей между каналами управления. Это объясняется тем, органы управления таких систем обычно связаны с самим объектом управления и эффект оказываемый ими на параметры объекта управления зависит от его текущего состояния.

Вместе с тем управление в скалярных каналах управления не являются естественными для объекта управления, поскольку построенная в терминах скалярных каналов система управления сильно зависит от выбора этих самых каналов и может работать только в ограниченном множестве режимов.

Для разрешение этой проблемы имеет смысл перейти к рассмотрению построения систем уравления в терминах объектов имеющих завершенный геометрически-физический смысл, то есть в терминах тензоров. Это не в коей мере не означает, что из уравнений стабилизации куда-либо денуться скалярные компоненты, но все они будут подчинены соответствующим тензорам, а следовательно выбор конкретной системы координат для расчета не будет влиять на динамику системы в целом.

\section{Требования к объектам управления.}
Забегая немного вперёд, можно сказать, что формализм тензорных сигналов удобен для построения систем автоматического управления объектами совершающими сложные движения в условиях пространства, мерность которого выше единицы.

К таким объектам относятся роботы-манипуляторы, дроны, некоторые виды автомашин, шагающие роботы. 

Данная статья посвящена математическому формализму обобщающему задачи управления этими и другими групами ОУ.

\section{Обработка тензорных сигналов. Общие положения и принципы.}
Тензорным сигналом будем называть изменяющуюся во времени геометрическую сущность, независимую от выбора системы координат, и однозначно представленную набором (или эквивалентными наборами) своих компонент. 

Особенностью тензорного сигнала является множественность его возможных представлений вплоть до того, что в различных частях системы управления один и тот же тензорный сигнал может быть представлен разным набором компонент (так например, тензор угловой ориентации может быть представлен матрицей поворота, кватернионом или вектором наименьшего поворота, а компоненты линейной скорости и всех прочих векторов будут зависить от текущего расчетного базиса). Выбор компонентного представления в вычислительной системе не влияет на динамические свойства системы, поскольку операции над тензорами эквивалентны во всех системах координат. Вместе с тем, для построения системы управления в терминах тензоров следует учитывать, что операции над тензорными сигналами производятся с учетом их геометрической природы и должны, при необходимости, выполнятся с правильными преобразованиями компонентных представлений (и базисов систем координат компонентных представлений). Поскольку не все тензорные уравнения являются линейными, для замыкания цепей систем управления следует использовать такие компонентные представления тензорных сигналов, которые дают наилучшую линейность с замыкаемыми по ним параметрами. Это позволяет добиться наилучшего качества управления и предсказуемости поведения системы.

В зависимости от природы тензорного сигнала, он может иметь разные алгебраические свойства. Тензорный сигнал считается линейным, если, он допускает выполнения операций сложения, вычитания, умножения на скаляр и линейный тензорный оператор. Для линейных тензорных сигналов однозначно определены тензор производной и тензор первообразной.

Другие типы тензорных сигналов могут иметь иные алгебраические свойства. Так, например, тензор ориентации не обладает свойством аддитивности, но, зато относительно него определена операция композиции. Композиция имеет некоторые свойства операции сложения, но не обладает свойством коммутативности.

Важное практическое значение в ТАУ имеет дифференцирование и интегрирование сигналов. Поскольку не все тензорные сигналы являются линейными и могут быть продифференцированы в компонентном представлении, для построения систем управления с тензорными сигналами введем понятие "линейной тензорной производной". 

Линейной производной тензорного сигнала $A(t)$ по параметру $t$ будем называть сигнал $B(t)$, для которого существует компонентное представление $b_i(t)$ численно равное локальной производной по параметру t некоего компонентного представления $a_i(t)$ и при этом не зависящее от компонент $a_i$. Такой сигнал также является тензором и существует как геометрическая сущность независимо от тензорного сигнала $A(t)$.

Важность данной операции объясняется тем, что, по построению, вне зависимости от текущих координат $a_i$, тензор $B$ в представлении $b_i$ всегда будет линеен по отношениюю к тензору $A$ в представлении $a_i$, а следовательно вне зависимости от выбранного режима будет одинакого входить в уравнения всех линеаризованных режимов.

Докажем, что тензор угловой скорости является линейной производной тензора ориентации в изложенном выше смысле.

Рассмотрим движущийся объект в мгновенном базисе собственной системы координат. Рассмотрим положение базиса через малый промежуток времени $\Delta t$. Запишем тензор ориентации объекта в компонентной форме через вектора кратчайшего поворота.

\begin{equation}H^j_i(t) = |\rho(t), r(t)| = |0, 0|\end{equation} 
\begin{equation}H^j_i(t+\Delta t) = |\rho(t + \Delta t), r(t + \Delta t)| = |\rho, r|\end{equation} 
\begin{equation}\lim_{\Delta t \to 0}\frac{H^j_i(t+\Delta t) - H^j_i(t)}{\Delta t} = \lim_{\Delta t \to 0}|\frac{\rho}{\Delta t}, \frac{r}{\Delta t}| = |\omega, v|\end{equation} , где $\omega$ и $v$ - мгновенные угловая и линейные скорости.

Суммирование компонент вектора поворота возможно в силу колинеарности нулевого вектора любому другому вектору и тому обстоятельству, что повороты по колинеарным векторам образуют группу.

Поскольку, вне зависимости от текущей ориентации объекта мы всегда можем записать его движение в собственном мгновенном базисе и получить его производную в форме независимой от параметров ориентации, тензор $\omega_i$ является линейной производной тензора ориентации.

Альтернативное доказательство можно провести из уравнения связывающего производную вектора поворота с вектором угловой скорости.

$$

Поскольку, в силу тензорной природы мы вольны выбирать систему отсчета, мы всегда можем выбрать её таким образом, чтобы мгновенное значеное $\rho$ в начальный момент времени равнялось нулю. Задав таким образом компонентное представление тензора ориентации, получаем линеаризованную производную в виде вектора угловой скорости.

Введение линеаризованной производной необходимо для того, чтобы иметь возможность проводить операцию дифференцирования тензора без указания конкретного компонентного представления и системы координат. 

Интегрирование тензорного сигнала.

Операция нахождения первообразной есть операция обратная нахождению производной сигнала. В случае если мы имеем дело с линейным векторным сигналом, операция поиска первообразной сводится к интегрированию компонет входного сигнала в каком-либо неизменном базисе, но в общем случае, для нелинейных сигналов, поиск первообразной есть нетривиальная операция, а смысл ее зависит от конкретной природы сигнала (как например, в случае с ориентацией и угловой скоростью, где первообразная определяется через композицию мгновенных поворотов). Невозможность построить первообразную без знания природы сигнала объяняется тем, что при нахождении линейной производной для нелинейных сигналов происходит потеря информации. Приближенное численное восстановление первообразной сигнала возможно по алгоритму приращение-коррекция. Однако, на практике, в ТАУ интегрирование сигналов происходит в области малых отклонений, тоесть в условии, когда приращения довольно точно передают характер поведения производной, а потому, вместо первообразной допустимо использовать интеграл компонентного представления входного тензорного сигнала. 

Такой интеграл будем называть интегралом тензорного сигнала по компонентному представлению или линеаризованным интегралом.

Следует учесть, что накапливаемый в вычислительной машине интеграл тензорного сигнала также является тензором, а потому при переходе компонентного представления в другую систему отсчета, интеграл должен пересчитываться соответственно. 

\section{Метод структурных схем в применении к тензорным сигналам.}
Обладая операциями линеаризованного дифференцирования и линеаризованного интегрирования тензорных сигналов, а также операциями сумирования, вычитания и умножения (на скаляр и матрицу), мы можем применить классический метод структурных схем к тензорным сигналам.

\section{Матричный коэффициент усиления сигнала}

\section{Анализ устойчивости}

\section{Группы органов управления}
Невозможность создать тензорное управляющее воздействие одним органом управления очевидно разрешается при использовании нескольких совместно действующих органов управления.

Группой органов управления G будем называть совокупность органов управления, совместно решающих задачу построения тензора управления U в виде $U = U(v_0, v_1, ..., v_n)$, где $n$ - количество органов управления. Если относительно управляющих воздействий органов управления выполняется принцип суперпозиции, то U является линейной комбинацией. 

$U = a_0v_0 + a_1v_1 + ... + a_iv_i.$
$u_i=a^i_jv^j$
$U = AV$

Уравнение () есть система линейных уравнений к решению которой сводится задача поиска управляюющего воздействия отдельных органов группы. Система () может иметь одно решение, не иметь решений вовсе или же иметь множество решений.

Случай отсутствия решений означает, что желаемое управление, требуемое от группы не может быть выполненно (вероятно, в силу физической несовместимости). 

Случай множества решений означает, что желаемое управление может быть достигнуто множеством способов. Поиск оптимального решение на данном множестве требует введения функционала оптимизации и, возможно дополнительных условий.

$F(V) -> min$
$AV = U$
$CV <= d$

Если F(V) - квадратичный функционал, а дополнительны условия отсутствуют, задача () является задачей квадратичного программирования.  

\section{Матрица чувствительности группы органов управления.}
Вернемся к вопросу поиска матрица A из уравнения ().

Как было сказано выше, уравнение в форме () может быть записано, если относительно воздействия органов управления выполняется принцип суперпозиции. 

Матрица A есть матрица частных производных $\frac{\partial{u^i}}{\partial{v^j}}$.

Практически значимыми примерами групп органов управления являются системы с суперпозицией силовых и мгновенных кинематических воздействий. Для них матрица А формируется из компонент линейного оператора переноса соответствующего воздействия:

Силовой перенос.
Кинематический перенос.

Из формы операторов переноса следует, что линейные и угловые параметры при построении желаемого управления должны рассматриваться совместно. Такой подход свойственнен для винтового исчисления / исчисления векторных моторов. Следует отметить, что векторный мотор является тензором и к нему применимо всё, описанное в предыдущих разделах.  

\section{Учёт неидеальности передаточных функций органов управления.}
В следствие неидеальности органов управления, в структурной схеме между вычислителем и эффектом управления возникает обусловленная динамикой органов управления передаточная функция. Особенностью групп органов управления является то, что в общем случае эта функция является диагонально матричной, поскольку каждый орган управления может иметь отличающиеся передаточную функции и постоянную времени. С целью улучшения качества управления следует стремиться выравнить передаточные функции органов управления. Если различия постоянных времени и передаточных функций обусловленно конструкционно, достигнуть этого эффекта можно с помощью введения дополнительной фильтрации управляющих сигналов по каждому отдельному каналу.

\section{Эффект миграции управляющего сигнала.}
Инерционность органов управления также приводит к эффекту миграции управляющего сигнала, выраженного в том, что в случае изменения оператора переноса управляющего воздействия (вследствии каких-либо эволюций системы, например её поворота или изменения плеча), управление накопленное органом управления неуспевает подстроиться под новую матрицу чувствительности и тензор управления получает неучтенную добавку.

Вычислим ее (учитывая, что R - линейный оператор):

$\dot{U_vi} = \dot{R\times V_i}$
$\dot{U_vi} = R \times \frac{\partial{V_i}}{t} + \dot{R} \times V_i$

Второе слагаемое в этом уравнении отвечает за неучтенный сигнал миграции управляющего воздействия.  

\section{Вывод.}

\end{document}

