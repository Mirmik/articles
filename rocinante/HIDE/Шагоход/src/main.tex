\documentclass[a4paper]{article}

\usepackage[12pt]{extsizes}
\usepackage[left=30mm, top=25mm, right=20mm, bottom=25mm, nohead]{geometry}

\usepackage{xcolor}

\usepackage{amssymb}
\usepackage{amsmath}
\usepackage{lipsum}
\setlength{\parindent}{3ex}
\setlength{\parskip}{1em}

\usepackage[utf8]{inputenc}
\usepackage[T2A]{fontenc}
\usepackage[english, russian]{babel}
\usepackage{caption}

\usepackage{graphicx} 
\graphicspath{ {./images/} }

\definecolor{shadecolor}{RGB}{230,230,230}

\begin{document}

% НАЧАЛО ТИТУЛЬНОГО ЛИСТА
\begin{center}
\hfill \break
\hfill\break
\hfill \break
\hfill \break
\hfill \break
\hfill \break
\hfill \break
\hfill \break
\hfill \break
\hfill \break
\large{<<Компенсация силовых невязок в тензорных системах управления.>>}\\
\hfill \break
\hfill \break
\hfill \break
\hfill \break
\hfill \break
\hfill \break
\hfill \break
\hfill \break
\end{center}
 
\normalsize{ 
\begin{tabular}{lccl}
Автор: Н.Ф. Сорокин\\\\
\end{tabular}
}\\
\hfill \break
\hfill \break
\hfill \break
\hfill \break
\hfill \break
\begin{center} \today \end{center}
\thispagestyle{empty} % выключаем отображение номера для этой страницы
\newpage
% КОНЕЦ ТИТУЛЬНОГО ЛИСТА

\section{Постановка задачи.}
Спроектируем систему управления шагающим механизмом на основе метода тензорных сигналов.

\end{document}

