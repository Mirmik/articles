\documentclass[a4paper]{article}

\usepackage[12pt]{extsizes}
\usepackage[left=30mm, top=25mm, right=20mm, bottom=25mm, nohead]{geometry}

\usepackage{xcolor}

\usepackage{amssymb}
\usepackage{amsmath}
\usepackage{lipsum}
\setlength{\parindent}{3ex}
\setlength{\parskip}{1em}

\usepackage[utf8]{inputenc}
\usepackage[T2A]{fontenc}
\usepackage[english, russian]{babel}
\usepackage{caption}

\usepackage{graphicx} 
\graphicspath{ {./images/} }

\definecolor{shadecolor}{RGB}{230,230,230}

\begin{document}

% НАЧАЛО ТИТУЛЬНОГО ЛИСТА
\begin{center}
\hfill \break
\hfill\break
\hfill \break
\hfill \break
\hfill \break
\hfill \break
\hfill \break
\hfill \break
\hfill \break
\large{Реферат.}\\
\large{<<Философия техники. Место техники в культуре от архаических времён до наших дней.>>}\\
\hfill \break
\hfill \break
\hfill \break
\hfill \break
\hfill \break
\hfill \break
\hfill \break
\hfill \break
\end{center}
 
\normalsize{ 
\hspace{100mm}Н.Ф. Сорокин
\hfill \break
\hfill \break
\hfill \break
\hfill \break
\hfill \break
\hfill \break
\begin{center} Москва 2019 \end{center}
\thispagestyle{empty} % выключаем отображение номера для этой страницы
\newpage
% КОНЕЦ ТИТУЛЬНОГО ЛИСТА

% Станица содержания.
\tableofcontents
\newpage

\section{Историческая ретроспектива.}

\subsection{Понимание техника архаическими культурами и культурами древних царств.}
Использование инструментов не уникально для человеческого вида. Многие животные используют и даже изготавливают инструменты, возводят, эксплуатируют и поддерживают в рабочем состоянии сложные конструкции.

Вне научного подхода инструмент не воспринимается чем-то специфическим: эффект его использования не имеет обоснования, как не имеют обоснования движения рук или ног. Мы делаем и всё. Мы просто знаем, что в такой-то ситуации мы можем применить такой-то инструмент и делаем это естественно без необходимости задумываться о законах физики, лежащих в основе совершаемых нами действий. 

Однако, со времен когнитивной революции и появления культуры, все стороны жизни были подвергнуты мифологическому осмыслению. Использование инструментов также, и даже, в первую очередь стало объектом этого процесса. Теперь использованием инструментов, предметов стали ведать духи, а удачные варианты применения ритуализировались и в таком виде передавались последующим поколениям.

Почему человек архаической культуры, скажем, может поднять (сдвинуть, переместить) огромный камень палкой, которая опирается на другой камень, хотя голыми руками он это сделать не в состоянии. Тур Хейердал в книге «Аку-Аку» описывает древнюю технику поднятия каменных идолов, весящих два-три десятка тонн. Под блок подводились три рычага, на которые по команде надавливали одиннадцать человек. При этом ещё один человек подсовывал под этот край блока камешки. Постепенно удавалось подвести камешки всё большего размера, и в результате блок медленно подымался на горке камней.

Древняя технология, описанная Хейердалом, весьма характерна для анимистических техник. Она включает: серию подсмотренных и отобранных в практике операций, обязательно предполагает ритуальные процедуры, передаётся в устной традиции из поколения в поколение. Когда Тур Хейердал спрашивал старосту, сохранившего по наследству от своего деда секрет подъёма и передвижения идолов, как они доставлялись из карьера и поднимались, то обычно получал такой ответ: «духи двигались сами», «они сами вставали». 

Спрашивается, как понимали люди анимистической культуры свои «технические» действия. Им, очевидно, не могло прийти в голову, что они могут заставить тотемного духа без его желания встать или идти. Другое дело — склонить его (жертвоприношением, просьбами и тому подобное) действовать в нужном для человека направлении. Когда староста объяснял Туру Хейердалу, что духи «сами встают и идут», он не имел в виду каменные скульптуры, речь шла именно о тотёмных духах. Сложные технические действия людей служили одной цели — побудить, заставить духов встать и идти. Когда архаический человек подмечал эффект какого-нибудь своего действия (удара камня, действия рычага, режущие или колющие эффекты), он объяснял этот эффект тем, что подобное действие благоприятно воздействует на души и духов.

Мировозрение человека архаической культуры не имело такого понятия, как техника, как такового. Действительно, разве непонятно, что духи направляют копьё, если согласно верного ритуала бросить его, направляют стрелу, если должным образом воспользоваться луком? Или, что духи помогут скрепить куски ткани, если правильно использовать иглу и нить? Поскольку всё происходит по воле духов, нет никакой необходимости видеть за рецептами/риталами какие-то основопологающие принципы и как-то объяснять происходящие. Это так, потому что так есть.

По ходу эволюции культуры, переходу к осёдлой жизни и увеличению роли инструментов и технических сооружений в жизни людей, мировозрение становится всё более и более механистическим. Появляются понятия естественного и искуственного. Теперь мир напоминает машину, отлажденную, запущенную и следующую некоему неизвестному плану. У этой машины появляются часовщики, которые занимаются поддержанием естественного хода вещей. На место духов становятся боги.

В культуре древних царств присутствием юогов и их участием объясняется действие технических вещей. Например, в Шумере боги вместе с людьми отвечают за «производственные» процедуры, так бог Солнца отвечает за дневной свет и тепло, Иш-тар, богиня луны — за ночное освещение, боги города — за городской порядок, бог кирпичей (был и такой бог в Шумере) отвечает за то, чтобы кирпичи имели правильную форму и быстро сохли.

В древней Греции, отмечает К. Хюбнер, «Афина Эргана является богиней ремесла, гончарного дела, ткачества, колесного дела, маслоделия и тому подобной деятельности. Горшечники обращаются к ней в своей песне, чтобы она простерла свою длань над гончарной печью, и свидетельствуют о присутствии богине в мастерской» Можно заметить, что деятельность богов в данном случае понимается не антропоморфно (захотел — освещаю, не захотел — не освещаю, захотел — кирпичи сохнут, не захотел — вообще никогда не высохнут), а скорее функционально. Функция бога кирпичей именно в том, чтобы кирпичи быстро сохли и имели правильную форму.

По сути, машинерия мира диктуется богами, а мастера и ремесленники - такие же механики мира, только рангом и могуществом и опытом поменьше. Мастера обращаются к богам за советам и поддержкой, поскольку те неизмеримо опытнее их в ремесле, а также надеются на то, что порядок вещей, поддерживаемых богами будет поддержан на надлежащем уровне.

Такое механистическое понимание какртины мира ставит вопрос о естественных процессах, достоверном знании, однажды установленном и существующем вечно. О чем-то что можно изучить, обосновать и структурировать. Именно механистика привела к постепенному становлению философии и науки, а философия со своим обобщающим подходом в дальнейшем укрепит позиции механистики. 

Однако это еще не та механистика, с которой мы имеем дело сегодня. В основе этой механистики лежат мифологические концепции. Доантичная и античная механистика исходит из того, что основе мира лежит нематериальное начало и старается разобраться в том, каким образом это нематериальное начало влияет на материальный мир. Поскольку все проявления материального должны являться продолжением мира нематериального, с переходом к античной культуре действие технических орудий и машин стало непонятным и потребовало объяснения. 

\subsection{Наука и осмысление техники в античные времена.}
Античное «технэ» — это не техника в нашем понимании, а все что сделано руками (и военная техника, и игрушки, и модели, и изделия ремесленников и даже произведения художников). В старой религиозно-мифологической традиции изготовление вещей понималось как совместное действие людей и Богов, причём именно Боги творили вещи, именно от божественных усилий и разума вещи получали свою сущность. В новой, научно-философской традиции ещё нужно было понять, что такое изготовление и действие вещей, ведь Боги в этом процессе уже не участвовали. Философы каждый день могли наблюдать, как ремесленники и художники создавали свои изделия, однако обычное для простого человека дело в плане философского осмысления было трудной проблемой.

Античная философия сделала предметом своего анализа прежде всего науку. Античные «начала» и «причины» — это не столько модели действительности, сколько нормы и способы построения достоверного (научного) знания. Соответственно, весь мир (и создание вещей в том числе) требовалось объяснить сквозь призму знания, познания и науки. У Платона есть любопытное рассуждение. Он говорит, что существуют три скамьи: идея («прообраз») скамьи, созданная самим Богом, копия этой идеи (скамья, созданная ремесленником) и копия копии — скамья, нарисованная живописцем.

Если для нашей культуры основная реальность — это скамья, созданная ремесленником, то для Платона — идея скамьи. И для остальных античных философов реальные вещи выступали не сами по себе, а в виде воплощений «начал» и «причин». Поэтому ремесленник (художник) не творил вещи (это была прерогатива Бога), а лишь выявлял в материале и своём искусстве то, что было заложено в природе.

Ещё труднее было объяснить действие орудий. В «Механических проблемах» (по видимому неверно приписываемых Аристотелю) действие весла, корабельного руля, мачты, паруса, пращи, щипцов для удаления зубов и так далее объясняется таинственными свойствами рычага и круга. Собственно никакого объяснения нет. Тем не менее, следует подчеркнуть, что именно Аристотель ввёл различение деятельности и природных явлений. Если деятельность предполагает усилия человека (цель и способности), то природные явления совершаются без его участия, сами по себе («Природа, говорит Аристотель — есть известное начало и причина движения и покоя для того, чему она присуща первично, по себе, а не по совпадению»).

Аристотель, как известно, отрицавший платоновскую концепцию идей, тем не менее, пытался понять, что такое создание вещей, исходя из предположения о том, что в этом процессе важная роль отводится познанию. В «Метафизике» мы встречаем такое рассуждение: если известно, что болезнь представляет собой то-то (например, «неравномерность»), а равномерность предполагает тепло, то, чтобы устранить болезнь необходимо нагревание.

Познание — это по Аристотелю движение в знаниях, а также рассуждение, которое позволяет найти последнее звено (в данном случае тепло), а практическое дело («создавание»), наоборот, — движение от последнего звена, опирающееся на знания и отношения, полученные в предшествующем рассуждении. Это и будет по Аристотелю создание вещи. Для современного сознания в этом рассуждении нет ничего особенного, все это достаточно очевидно. Не так обстояло дело в античные времена. Связь деятельности по созданию вещей с мышлением и знаниями была не только не очевидна, но, напротив, противоестественна. Действие — это одно, а знание — другое. Потребовался гений Аристотеля, чтобы соединить эти две реальности.

Анализ античной практики, которая стала ориентироваться на аристотелевское решение и конструкцию практического действия, показывает, что были, по меньшей мере, две области, где знания отношений, полученных в научном рассуждении, действительно, позволяют найти это последнее звено и затем построить практическое действие, дающее нужный эффект. Это были практика изготовления орудий, основанных на действии рычага, и определение устойчивости кораблей в кораблестроении.

Так, Архимед, опираясь на закон рычага (который он сам вывел), определял при заданной длине плеч и одной силе, другую силу, то есть вес, который рычаг мог поднять (или при заданных остальных элементах определял длину плеча). Сходным образом (то есть когда при одних заданных величинах высчитывались другие) Архимед определял центр тяжести и устойчивость кораблей.

Подавляющая же масса античных техников действовали по старинке, то есть рецептурно, большинство из них охотнее обращались не к философии, а к магическим трактатам, в которых они находили принципы, вдохновляющие их в практической деятельности. Например, такие: «Одна стихия радуется другой», «Одна стихия правит другой», «Одна стихия побеждает другую», «Как зерно порождает зерно, а человек человека, так и золото приносит золото». По происхождению эти принципы имели явно мифологическую природу (пришли из архаической культуры), однако в античной и средневековой культурах им был придан более научный (естественный) или рациональный (рецептурный), характер. Поэтому речь идёт уже не о духах или богах и их взаимоотношениях, а о стихиях, их родстве или антипатиях, о якобы естественных превращениях.

\subsection{Зарождение современного технического мировозрения.}
И в Средние века техника всё еще осмыслялась не как техника в нашем понимании. Мастер, всего лишь, подражал Творцу, техника понималась как мистический процесс, позволяющей мёртвое сделать живым (поскольку произведение понималось как живая вещь, субъект). «Так и гончар, — пишет Тертуллиан, — способен, воздействуя огнём, сгущать глину в твёрдую массу и из одной формы делать другую, лучше прежней, уже особого рода со своим собственным именем. Такое произведение называют бессмертным, потому что оно образовалось «жаром Божественного дыхания»

Cредневековая техника — это мистический процесс создания живых вещей, предполагающий подражание Творцу и любовь.

Сложности в объяснении техники не должны, однако, закрывать от нас кардинальных изменений, подготовивших к концу Средних веков и эпохе Возрождения предпосылки как естествознания, так и инженерии. Именно в этот период формируется новое понятие природы, как бесконечного источника сил и энергий (сначала божественных, потом естественных). 

С точки зрения христианского мировоззрения природа создана для человека, который сам создан «по образу и подобию» Бога, то есть обладает разумом, отчасти, сходным с божественным. Поэтому человек при определённых духовных условиях в состоянии приобщиться к замыслам Бога, в результате он может узнать устройство и план природы, замыслы и законы, в соответствии с которыми происходят природные изменения.

Что можно извлечь из этих исторических реконструкций? Судя по всему, техника всегда понималась не просто как мёртвое образование, а как живая сущность (душа, дух, боги, «жар божественного дыхания»), которые помогают человеку, если только он действует правильно по отношении к этой сущности (приносит ей жертву, славит, проникается её намерениями, но также прикладывает мускульные усилия, использует орудия, действует по определённой логике, создаёт произведение — все это тоже есть момент сакрального действия). 

Следует отметить, что до последнего времени техника не была столпом культуры. До появления естественных наук и инженерии возможность обнаруживать эффекты природы и управлять природными процессами были ограничены техническим опытом человека, который основывался на методе проб и ошибок. Культура же, представляющая собой социальный организм (культуры складываются, проходят цикл жизни и развития, а по истечении времени умирают, уступая место следующим культурам), не может строиться на таком ненадёжном источнике. Представления о духах, богах, сущем и Творце были достаточны, чтобы возникли и развивались соответствующие культуры, но недостаточны, чтобы на регулярной основе обнаруживать новые природные процессы и эффекты и овладевать ими, превращая их в средства человеческой деятельности.

Совершенно другая картина складывается в культуре нового времени. Пройдя период переосмысления в Средние века, природа стала пониматься как источник скрытых сил и энергий, которыми человек может овладеть, если только в новой науке он выявит устройство (законы) природы. В результате на рубеже XVI–XVII веков формируется своеобразный социальный проект — создание новых наук и овладение силами природы с целью преодоления кризиса и установления в мире нового порядка, обеспечивающего человеку почти божественное могущество.

Исследования Галилея и Гюйгенса позволили создать предпосылки для реализации этого проекта. Именно Галилей показал, как строить новые науки. Для этого теорию необходимо было ориентировать на обслуживание техники, а полученные в ней теоретические характеристики изучаемых природных явлений относить не к этим явлениям, а к процессам в эксперименте, последние техническим путём приводились в соответствие с математическими построениями самой теории. В результате теория позволяла предсказывать поведение изученных природных явлений, то есть по отношению к техническому действию выступала как модель.

Любопытно, что моделирование мира, как таковое - не есть изобретение науки.

Колдуны и шаманы часто прибегают к приёму олицетворения, очеловечевания предметов и явлений природы. Отождествления могут строится между предметами (символический меч), между процессами (погода варится в котле). Приёмы отождествления оформляются в целые языки, рунические письмена, карты таро. Каждый символ имеет свой смысл и отражение в реальном мире, предмет или явление, с которым образ отождествлен, то есть модель объекта.

Королем всех языком моделирования объектов внешнего мира, несомненно, является математика, и в этом плане она имеет более чем глубокие корни. Высшая магическая нотация, алгебра, позволяет моделировать предметы и явления с ужасающей непосвященных точностью.

Создавая модели предметов, явлений и даже самой себя, математика вычленяет наиболее важные, существенные свойства объекта и описывает их языком символов, отбрасывая несущественное. Записав уравнение, шаман, водя пером повинуется смыслу уже начертанных символов, записывая следующие, а после читает значение открывшегося текста, совершая ритуал открытия нового знания.

Зародившийся как средство ведения подсчетов собранных урожаев и бугалтерских расчетов, аппарат математики с поразительной скоростью внедрился в культурный фундамент человечества, оплетя бедных ничего неподозревающих обывателей вереницей рун, абстрактных, опастных для незакалённого ума понятий и концепций, навсегда подчинив себе наш вид. Благодаря математике люди создают противоестественные конструкции, раскрывают сокровенное и даже провидят будущее.

Тексты магических рун, программы, не просто приводят в движение механических големов, но, благодаря могуществу магического языка, делают голема послушным воле создателя. Колонки цифр и формул становятся действием.

Открытие экспериментальной механики позволило сформировать законы механики на языке математики и перейти к решению обратной задачи, а именно конструированию процессов на основе знаний о законах механики.

Так Гюйгенс сконструировал часы с изохронным качанием маятника, то есть подчиняющимся определённому физическому соотношению (время падения такого маятника от какой-либо точки пути до самой его низкой точки не должно зависеть от высоты падения). Анализируя движение тела, удовлетворяющее такому соотношению, Гюйгенс приходит к выводу, что маятник будет двигаться изохронно, если будет падать по циклоиде, обращённой вершиной вниз. Открыв далее, «что развёртка циклоиды есть также циклоида», он подвесил маятник на нитке и поместил по обеим её сторонам циклоидально-изогнутые полосы так, «чтобы при качании нить с обеих сторон прилегала к кривым поверхностям. Тогда маятник действительно описывал циклоиду».

Таким образом, исходя из технического требования, предъявленного к функционированию маятника, и знаний механики, Гюйгенс определил конструкцию, которая может удовлетворять данному требованию. Решая эту техническую задачу, он отказывается от традиционного метода проб и ошибок, типичного для античной и средневековой технической деятельности, и обращается к науке. Гюйгенс сводит действия отдельных частей механизма часов к естественным процессам и закономерностям и затем, теоретически описав их, использует полученные знания для определения конструктивных характеристик нового механизма.

Тем самым Гюйгенс практически осуществил то целенаправленное применение научных знаний, которое и составляет основу инженерного мышления и деятельности. Для инженера всякий объект, относительно которого стоит техническая задача, выступает, с одной стороны, как явление природы, подчиняющееся естественным законам, а с другой — как орудие, механизм, машина, сооружение, которые необходимо построить искусственным путём. Сочетание в инженерной деятельности «естественной» и «искусственной»: ориентации заставляет инженера опираться и на науку, из которой он черпает знания о естественных процессах, и на существующую технику, где он заимствует знания о материалах, конструкциях, их технических свойствах, способах изготовления и так далее.

Дихотомия теории и практики, по видимому является отголоском разделения мира на идеальное и материальное, свойственное для античной и доантичной философии. Современное положение дел перекликается с размышления Платона об лежащей в основе всего Идее. Исследование идеального получило самое широкое развитие и применение. Ни одно серьёзное проектирование ни обходится без построения идеальной модели и поиска закономерностей её работы.

Итак, если Галилей создал первый образец естествознания, то Гюйгенс — инженерного действия, то есть показал, как на основе знаний новой науки (позднее она получила названия «естественной») создавать технику, где бы, во-первых, реализовались уже изученные в естественной науки процессы природы, во-вторых, ими можно было управлять.

Как следствие, постепенно формируется мировоззрение, что «природа написана на языке математике», представляет собой скрытый механизм, однако, в естественной науке этот скрытый механизм можно описать в форме законов природы, а в инженерии, используя эти законы, создавать реальные механизмы.

За прошедшие с тех пор три-четыре сотни лет, научное мировозрение окрепло и пустило корни.

Исходной предпосылкой технократического дискурса является убеждение в том, что современный мир — это мир технический и что техника представляет собой систему средств, позволяющих решать основные цивилизационные проблемы и задачи, не исключая и тех, которые порождены самой техникой. В. Рачков в прекрасной книге «Техника и её роль в судьбах человечества», посвящённой преимущественно анализу и критике технократического дискурса, пишет: «Самым модным и расхожим тезисом сегодня является: отныне все зависит от техники, поскольку, несомненно, мы находимся в обществе, созданном целиком техникой и для техники… Как только человек осознает какую-то проблему или опасность, так сразу же можно сказать, что он берётся за её рассмотрение и решение, и, можно сказать, что она уже потенциально разрешена.

Иначе говоря, существует негласная установка, что каждое затруднение нашего мира, если на него выделяется достаточно технических средств, людских и денежных ресурсов, преодолевается по мере того, как за него принимаются всерьёз. Более того, любое достижение в области науки и техники призвано решать определённое число проблем. Или, точнее, перед лицом опасности, конкретной, лимитированной трудности, люди обнаруживают неизбежно адекватное техническое решение.

\section{Эволюция понимания техники.}

Истоки философии техники прослеживаются в трудах древних философов, но систематическое философское исследование феномена техники началось в лишь конце ХIХ — начале ХХ века. Термин «философия техники» в научный обиход ввёл немецкий учёный Эрнест Капп, в 1877 году выпустивший книгу «Основные линии философии техники».

Философия, конечно же не могла обойти стороной такой вопрос, как "что есть техника", и как наша способность использовать инструменты встраивается в архитектуру мира, тщательно моделируемую философами.

Конечно, приведенные выкладки относятся к ситуации в целом. Моряки и кораблестроители использовали математику и навигацию задолго до формального математического обоснования корректности их методов и проводили при этом довольно сложные расчеты. Однако, это не означает, что мастера корабелы имели научное мировозрение.

Если мы взглянем на историческую ретроспективу, мы увидим, что становление понятие техники было постепенным, эволюционировало сравнительно медленно и естественным образом перекликалось с изменением жизненного уклада людей.

Если в анимистической культуре техника вообще никак не обособлялась от остальной реальности, то с появлением поселений и, тем более, частной собственности и торговли, вещи стали делить на естественные и искуственные, созданные человеком.

Постепенная смена анимизма на политеизм, накопления знаний о мире, земледелие и необходимость следить за сменой сезонов изменила взгляд людей на устройство мира, заставив задуматься о существовании верховного порядка, основного процесса, приводящего в движения небесные тела, воду и ветра, руководящего ростом растений и жизнью прочих живых существ.

Желание постичь этот порядок породило философию, которая в поиске создала во множестве различные концепции устройства мира, мимоходом установив, что теоретические построения - это не просто гимнастика для мозга, но они могут быть с пользой применены на практике. 

Многие построения тех времен не выдержали проверку временем. Вообще для античной и смежной с античной философии типично поставить во главу угла некое красивое умозрительное построение и пытаться вывести из него существование всех прочих вещей.

Во времена между античной эпохой и эпохой возрождения этот подход был пересмотрен. Концепция эксперимента и использование математики в фундаментальных расчётах позволило не только научиться делать верные предсказания о работе механизмов, но и конструировать механизмы с заранее заданными свойствами.

Такое положение дел увеличило доверее человека к технике. Люди стали значительно больше расчитывать на технические приспособления, что в конечном итоге, сделало технику одним из столпов человеческой культуры. 

Надо отметить, что наука породнила многие дисциплины. Медицина, которая во времена Парацельса также стала принимать черты строгой научной дисциплины обнаружила, что многие построения инженеров-механиков следует изучать для лучшего понимания анатомии. Сегодня мы также наблюдаем и обратный процесс. 

Мы видим, что изучение анатомии и функционирования живых систем позволяет нам строить эффективные механизмы, поскольку природа имела очень много времени на создание эффективных конструкций. Мы видим это сегодня, когда строим вычислительные системы по принципу нейронных сетей. Мы видели это сто лет назад, когда создали самолёт по принципу и подобию крыла птицы.

Концепции искуственного и естественного, в рамках которых человечество видит мир сегодня постепенно стираются. Если раньше техника была про исскуственное, то сейчас многие естественные процессы описываются в тех же самых терминах, что используются в технических системах, и появляются системы, прекрасно работающие с двумя этими ипостасями сразу.

Таким образом можно предположить, что в скором времени понятия искуственного перестанет противопостовляться понятию естественного, расширив тем сам мировозренческую область применения техники.

\section{Выводы.}
На основе анализа истории можно заключить, что процесс становления науки и техники есть процесс эволюционный и процесс этот еще не закончен. 

Каждый последующий этап эволюции понимания техники есть видоизмененный предыдущий, причем этот груз традиции накапливается и всё еще присутствует в современной технической культуре. Даже сегодня работа со многими механизмами выполняется методами проведения ритуалов, доставшихся нам от предков.

Практика показывает, что понимание развития техники, философии отдельных её дисциплин позволяет значительно лучше решать повседневные технические задачи. Значительно проще работать с системой, понимая образ мысли людей её конструировавших. 

Понимание вопроса о том, что есть практическая философия техники может быть достигнуто в процессе изучения философии технических дисциплин, культур социальных групп представляющих эту дисциплину специалистов. Как без понимания технической культуры невозможно естественное использование инструментов поржденных этой средой, так и использование научного метода затруднено в отрыве от понимания истории науки и техники, истории слагающих эти методы идей.

\newpage
\section{Список литературы:}
\begin{itemize}
\item Вадим Маркович Розин: Философия техники. 10.08.2013
\item Записи телеграмм канала Homo Mechanica.
\item К. Н. Хабибуллин, В. Б. Коробов, А. А. Луговой, А. В. Тонконогов. Философия науки и техники. Цикл лекций для адъюнктов и аспирантов. — М., 2008. 
\end{itemize}

\end{document}

