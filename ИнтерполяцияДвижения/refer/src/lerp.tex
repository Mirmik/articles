\section{Построение масива траекторных точек для решения задачи линейной интерполяции.}

Многие практические задачи, связанные с применением роботов манипуляторов, позиционеров и прочих подобных изделий не требуют синтеза управления в реальном времени. Вместо этого используется линейная интерполяция по заранее заданному массиву геометрических точек, между которыми манипулятор перемещается задавая постоянную скорость управления по кинематическим координатам. Этот массив или просчитывается незадолго до начала движения, либо вообще формируется один раз для всех последующих циклов работы механизма, если задача может быть описана циклом.

Несмотря на то, что целью настоящего метода не является построение интерполяционных массивов, мы можем использовать его для этой цели.

Разберём построение массива точек для прохода по траектории. Зададим положение и ориентацию управляемого звена как функцию времени $P(t)$. Прогоним математическую модель системы управления по этой траектории и с заданным дискретом (по времени или расстоянию) зафиксируем компоненты вектора $q(t_i)$, $i=0..n$. Массив векторов $q(t_i)$ есть искомый массив точек.

Компоненты массива могут содержать небольшую ошибку позиционирования, которую можно выбрать методом координатного спуска или запустив модель в режими стабилизации к соответствующему моменту $t$ траекторному положению.

Дальнейший проход по заданному таким образом массиву точек в общем случае имеет ошибку на интерполяционных участках. Для ее минимизации следует уменьшать интервал.