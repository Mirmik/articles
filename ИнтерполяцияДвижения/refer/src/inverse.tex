\section{Обратная скоростная задача.}\label{invspd_sect}

Имея решение прямой скоростной задачи и задавшись целевым 6-вектором скорости $V_s^*$ построим управление $\dot{q}^{*}$, удовлетворяющее тождеству
\begin{equation}\label{inverse_1}
T^* = V_s^* = \textbf{J}_s\dot{q^*}  
\end{equation}

Здесь $\textbf{J}_s$ - это часть компонентного представления тензора $J$, соответствующая звену $s$. $\textbf{J}_s$ представляет собой матрицу размерности $N\times6$, где N - количество обобщенных координат. Количество столбцов $6$ соответствует количеству компонент 6-вектора скорости. 

\colorbox{shadecolor}
{\parbox{0.9\textwidth}{В разделе \ref{restr_traj} показано, что метод позволяет строить управление по большему числу параметров, и что вектор $T^*$ может иметь более сложный смысл, нежели 6-вектор скорости звена. В настоящем разделе без потери общности рассматривается построение упровления по 6 компонентам 6-вектора скорости выходного звена позиционера.}}

Система линейных уравнений (\ref{inverse_1}) может иметь одно решение, иметь множество решений и не иметь решений вовсе. Случай отсутствия решений соответствует ситуации, когда позиционер физически не может выполнить перемещение в заданном направлении или в следствии геометрических ограничений, или в следствии нахождения в сингулярном к желаемому направлению положении. Методы недопущения подобных ситуаций обсуждаются в разделе \ref{restr_traj} и в соответствии с концепцией декомпозиции их избежание не является задачей координатного решателя. 

Рассмотрим методы решения задачи, при наличии как минимум одного решения.

\subsection{Метод координатного спуска.}

\subsection{Метод обратной матрицы}
