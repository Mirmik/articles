\section{Обратная скоростная задача.}\label{invspd_sect}

Имея решение прямой скоростной задачи и задавшись целевым 6-вектором скорости $V_s^*$ построим управление $\dot{q}^{*}$, удовлетворяющее тождеству
\begin{equation}\label{inverse_01}
T = V_s^* = \textbf{J}_s\dot{q^*}  
\end{equation}
\begin{equation}\label{inverse_02}
T = \sum_{i=0}^nS_i\dot{q^*_i}  
\end{equation}

Здесь $\textbf{J}_s$ - это часть компонентного представления матрицы \textbf{$J$}, соответствующая обобщенной координате $s$. $S_i$ представляет собой 6-вектор из которых образуется матрица $\textbf{J}_s$. Количество строк $6$ соответствует количеству компонент 6-вектора скорости. Назовём вектор $S_i$ вектором чуствительности по координате $i$.

\colorbox{shadecolor}
{\parbox{0.9\textwidth}{В разделе \ref{restr_traj} показано, что метод позволяет строить управление по большему числу параметров, и что вектор $T^*$ может иметь более сложный смысл, нежели 6-вектор скорости звена. В настоящем разделе без потери общности рассматривается построение управления по 6 компонентам 6-вектора скорости выходного звена позиционера.}}

Система линейных уравнений (\ref{inverse_02}) может иметь одно решение, иметь множество решений и не иметь решений вовсе. Случай отсутствия решений соответствует ситуации, когда позиционер физически не может выполнить перемещение в заданном направлении или в следствии геометрических ограничений, или в следствии нахождения в сингулярном к желаемому направлению положении. Методы недопущения подобных ситуаций обсуждаются в разделе \ref{restr_traj} и в соответствии с концепцией декомпозиции их избежание не является задачей координатного решателя. 

В целом следует отметить, что задача поиска линейной комбинации с учетом наложения условий на максимальную и минимальную скорости, а также при введении критерия минимизации длины вектора управления может быть рассмотрена как задача выпуклого программирования.

Рассмотрим методы решения задачи, при наличии как минимум одного решения.

\subsection{Метод координатного спуска / метод градиентного спуска.}

В этой категории методов поиск решения уравнения выполняется постепенным изменением вектора выходных координат $q$ в направлении соответствующем скалярному произведению векторов $S^i$ ($i=1..n$) на вектор текущей ошибки.

В рамках системы данная категория методов удобна, поскольку методы допускают вольное изменение весов отдельных координат, варьирование порядка, внедрение функционала барьерных функций для гибкого избежания коллизий (см. \ref{restr_traj}).

С другой стороны итерационная природа методов приводит к недетерминированности времени вычисления. Время вычисления и сходимость метода зависят от выбранного способа спуска и настроечных коэффициентов алгоритма. 

В силу итерационной природы использование метода в архитектуре основанной на ПЛИС или нейросетях возможно только при строго заданном количестве итераций.

\subsection{Метод псевдообратной матрицы}

СЛАУ (\ref{inverse_02}) может быть решена через поиск псевдообратной матрицы в виде
\begin{equation}\label{inverse_pseudo}
\dot{q^*} = \textbf{J}_s^+ T 
\end{equation}
Поиск псевдообратной матрицы при этом выполняется численно на каждой итерации и занимает достаточно малое время.

Недостатком метода псевдообратной матрицы является низкая по сравнению с методом градиентного спуска модифицируемость алгоритма, что осложняет внесения барьерных функций в алгоритм.

Возможность добавления весов скоростей координат зависит от выбранного метода ортогонального разложения, используемого при поиске псевдообратной матрицы. 
