\documentclass[12pt,a4paper,titlepage]{article}
\usepackage[left=30mm, top=25mm, right=20mm, bottom=25mm, nohead]{geometry}

\usepackage{xcolor}

\usepackage{amssymb}
\usepackage{amsmath}
\usepackage{lipsum}
\setlength{\parindent}{3ex}
\setlength{\parskip}{1em}

\usepackage[utf8]{inputenc}
\usepackage[T2A]{fontenc}
\usepackage[english, russian]{babel}
\usepackage{caption}

\usepackage{graphicx} 
\graphicspath{ {./images/} }

\definecolor{shadecolor}{RGB}{240,240,240}

\begin{document}
\date{27 ноября 2017}

\title{Тензорная следящая система}
\author{Сорокин Н.Ф.}

\maketitle
\newpage

\section{Введение.}

В настоящее время бурное развитие получили робототехнические системы, манипуляторы, позиционеры, станки ЧПУ, выполняющие задачу изменения положения рабочего органа (инструмента, схвата, зонда) в лабораторной системе координат и характеризыющиеся цепной компоновкой кинематических звеньев.

Обратная задача кинематики, решение которой требуется для управления такими изделиями нелинейна, может иметь множество решений и представлять значительную вычислительную сложность. В связи с этим было разработано множество методов решения обратной задачи кинематики.

Часть методов, в особенности аналитических, построена для решения частных случаев задачи управления кинематической цепью и может быть применена только к ограниченному числу изделий. К числу этой категории методов также относятся методы, работающие только с одним типом кинематических звеньев (в основном с одностепенными поворотными звеньями).

Наиболее общее, не зависящее от геометрии решение даёт применение матрицы Якоби, описывающей связь положения выходного звена с кинематическими координатами в частных производных.

Само по себе использование матрицы Якоби для расчета управления звеном кинематической цепи может быть не слишком удобным из-за неинтуитивности частных производных компонент тензора положения. В значительной степени неудобство управления в частных производных координат положения происходит из неочевидности физического смысла производных компонент объекта ориентации, который может задаваться матрицей размерности 3x3, или кватернионом, или быть частью более сложного алгебраического объекта, определяющего положение звена целиком, а именно матрицы размерности 4x4 или бикватерниона. Это ухудшает прозрачность алгоритма и увеличивает требования к квалификации разработчика програмного обеспечения конкретного изделия.

Для настоящего метода выбран скоростной аналог матрицы Якоби, характеризующий связь производных кинематических параметров с векторами линейной и угловой скоростей звеньев манипулятора. В настоящей работе показано, как решение скоростной задачи может быть использовано для координатно-траекторного управления широким классом открытых кинематических цепей. На основе метода приводится вариант алгоритма, достаточный для построения на его основе следящих систем, систем линейной интерполяции, а также достаточно простой для реализации в ЦПУ, ПЛИС, для управления реальным физическим объектом или моделью, а также обсуждаются варианты его модификации.

Следует отметить, что метод не налагает ограничения на класс кинематических пар и, хотя и разрабатывался для кинематических пар пятого класса, может быть применен для управления любой открытой кинематической цепью из простых, а с некоторыми доработками и из сложных звеньев.

\newpage
\section{Объект управления.}

Рассмотрим открытую кинематическую цепь состоящей из простых кинематических звеньев. С каждым кинематическим звеном свяжем две локальные системы координат, распожив их в точках сочленения кинематических пар. Таким образом каждое кинематическое звено имеет входную и выходную ЛСК, жестко связанные с ним.

В сочленении кинематической пары также оказывается две системы координат - по одной для каждого звена пары (выходная первого звена и входная второго). Введем кинематические параметры звеньев $q_i$, определяющих взаимное расположение локальных систем координат.

\begin{center}
  \includegraphics[width=0.6\textwidth,height=0.6\textheight,keepaspectratio]{ax3view.jpg}
  \captionof{figure}{Взаимное расположение систем координат.}
  \label{}
\end{center}

Таким образом кинематическая цепь определяется цепочкой локальных систем координат, положение каждой из которых определяется положением предыдущей и объектом гомогенного преобразования, либо константным, либо завищим от соответствующего кинематического параметра.

\colorbox{shadecolor}
{\parbox{0.9\textwidth}{Здесь и далее под объектом гомогенного преобразования (англ. homogenous transform) понимается геометрическая сущность, описывающая однородное координатное преобразование, состоящее из комбинации поворота и переноса. В разных частях настоящего изложения объект гомогенного преобразования фигурирует как оператор преобразования СК, способ описания положения кинематического звена имеющий смысл соответствующего тензора, или програмный объект (в терминах ООП), непосредственно используемый в програмном обеспечении. Подробнее в разделе \ref{geom}}}

Пронумеруем звенья и системы координат. Пусть неподвижное кинематическое звено, жестко связанное с лабораторной системой координат считается 0-ым, лабораторная система координат считается 0-ой СК, пусть входная система каждого $n$-ого звена считается $n$-ой СК. Выходную систему кинематического звена обозначим как $n^*$. 

Если в нашей цепи $m$ звеньев, то СК выходного звена цепи будет $m$-ой системой координат. (рассматривать $m^*$ для выходного звена цепи не имеет особого смысла).

Конкретный тип кинематического звена не имеет существенного значения, но желательно, чтобы тензор гомогенного преобразования, связывающий

\newpage
\section{Гомогенное преобразование. Тензор положения. Тензор скорости.}\label{geom}

Остановимся подробнее на введенном в рассмотрение объекте гомогенного преобразования, математическое представление которого может быть различным. В частности, для него могут применяться матрицы размерностью 4x4, бикватернионы, комбинация кватерниона поворота и вектора трансляции и т.д. В рамках рассматриваемого метода конкретная форма представления объекта взаимозаменяема и определяется исключительно удобством реализации в вычислительной системе. Обозначим объект гомогенного преобразования и тензор преобразования, выполняемого объектом как $H^j_i$. 

В програмной реализации удобно представлять $H^j_i$ как комбинацию двух компонент, кватерниона поворота $Q$ и вектора трансляции $S$, но, как сказано ранее есть и иные возможные представления. 

На множестве $H^j_i$ определена операция некомутативная операция композиции, часто обозначаемая как умножение. Множество $H^j_i$ замкнуто по этой операции.

Введем в рассмотрение объект положения СК и связанный с ним тензор положения. Обозначим его P. Объект положения удобно задавать тем же набором компонент, что и объект гомогенного преобразования. В такой нотации компонентное выражение тензора положения $P$ j-ой системы координат взятого в i-том базисе, будет совпадать с компонентным выражением тензора преобразования СК, из базиза i-той СК в базис связанный с j-ой СК. 

\begin{equation}
P_j^{(i)} = H_{ij}^{(i)}
\end{equation}

Также введем в рассмотрение производную тензора положения j-ой системы координат

\begin{equation}\label{speed_eq} 
V_j(t) = P_j'(t) 
\end{equation}

Как и тензор преобразования системы координат и тензор положения, производная тензора положения является геометрическим объектом и имеет смысл скорости изменения геометрического положения объекта или системы координат. Часто в качестве компонентного представления производной тензора представления выбирают производные компонент тензора положения, однако такой подход не является интуитивным и не очень удобен в вычислительной реализации. В настоящем методе в качестве компонентного представления производной тензора положения выбирается пара векторов его линейной и угловой скорости $(v,\omega)$

\begin{equation}\label{speed_eq_comp} 
V_j(t) = \dot{P}_j(t) = (\omega_j(t),v_j(t)) 
\end{equation}

При необходимости, корректность такого перехода может быть выведена аналитически в нотации бикватерионов. 

Представление (\ref{speed_eq}), удобно, в рамках метода, с вычислительной точки зрения. Следует учесть, что использование такой системы компонентного представления приводит к тому, что уравнение (\ref{speed_eq}) не может быть в общем случае записано в компонентной форме.

\colorbox{shadecolor}
{\parbox{0.9\textwidth}{Вектор, состоящий из компонент пары векторов $(\omega, v)$ далее называется 6-вектором скорости кинематического звена. Альтернативное представление тензора положения в виде пары радиус вектора и вектора поворота $(r,\rho)$ далее называется 6-вектором положения.}}

Одной из возможных форм компонентного представления тензора положения является пара радиус-вектора и вектора поворота $(r,\rho)$ (порядок преобразований - сначала поворот, затем трансляция). Отметим, что, если на 6-вектор скорости $(v,\omega)$ наложить условие колинеарности 6-вектору положения $(r,\rho)$, задача может быть сведена к одномерной, дифференциальные уравнения становятся линейными, а композиции преобразований по этой 6-оси становятся комутативными и образуют группу. В форме 3-векторов для этого должны выполняться соотношения.

\begin{equation}\label{} 
\bar{v} \upuparrows \bar{r}, \  \bar{\omega} \upuparrows \bar{\rho}, \  \frac{|\bar{v}|}{|\bar{r}|} = \frac{|\bar{\omega}|}{|\bar{\rho}|} 
\end{equation}

\begin{center}
  \includegraphics[width=0.8\textwidth,height=0.8\textheight,keepaspectratio]{oneaxis.png}
  \captionof{figure}{Сведение задачи к одномерному случаю.}
  \label{}
\end{center}

\begin{equation}
(\bar{v_l},\bar{\omega_l}) = (\dot{\bar{r_l}},\dot{\bar{\rho_l}})
\end{equation}

Это замечание пригодится для дальнейшего изложения и может быть полезно при анализе устойчивости системы управления в терминах объектов гомогенных преобразований. 

Хочется отметить, что анализ устойчивости систем управления в терминах алгебраических объектов, отличных от действительных чисел, таких как объекты гомогенных трансформаций может иметь существенное практическое значение и представляет интересное направление исследований.

\newpage
\section{Прямая скоростная задача.}

Рассмотрим производную тензора положения $j$-ой СК в цепочке, состоящей из $n$ звеньев.
\begin{equation}\label{eq1}
V^j = \frac{dP_j}{dt} = \sum_{i=1}^{n}\frac{\partial{P_j}}{\partial{q_i}}\dot{q_i} 
\end{equation}

Поскольку от $q_i$ напрямую зависит только $P_i$, а вариации остальных $P_j, j \neq i$ являются производными,
\begin{equation}\label{}
\frac{\partial{P_j}}{\partial{q_i}} = \frac{\partial{P_j}}{\partial{P_i}}\frac{\partial{P_i}}{\partial{q_i}}
\end{equation}

Причем, очевидно, что 
\begin{equation}\label{}
\frac{\partial{P_j}}{\partial{P_i}} = 0, \ \ \forall i: i > j
\end{equation}
, поскольку эволюции последующих звеньев не могут влиять на положение предшествующих.
\begin{equation}\label{}
\frac{\partial{P_j}}{\partial{P_i}} = \frac{\partial{P_j}\partial{t}}{\partial{P_i}\partial{t}} = \frac{\partial{\dot{P}_j}}{\partial{\dot{P}_i}}
\end{equation}
\begin{equation}\label{}
\frac{\partial{P_i}}{\partial{q_i}} = \frac{\partial{P_i}\partial{t}}{\partial{q_i}\partial{t}} = \frac{\partial{\dot{P}_i}}{\partial{\dot{q}_i}}
\end{equation}

Тогда (\ref{eq1}) можно записать в тензорном виде как:
\begin{equation}\label{eq2}
V^j = \frac{\partial{\dot{P^j}}}{\partial{\dot{P}^i}}\frac{\partial{\dot{P}^i}}{\partial{\dot{q}^i}}\dot{q}^i 
\end{equation}

Переходя к записи в интересующих нас компонентах скоростей имеем:
\begin{equation}\label{matrix_spd_eq}
\begin{vmatrix}
\omega^j\\
v^j
\end{vmatrix}
=
\begin{vmatrix}
\frac{\partial{\omega^j}}{\partial{\omega^i}} & \frac{\partial{\omega^j}}{\partial{v^i}} \\
\frac{\partial{v^j}}{\partial{\omega^i}} & \frac{\partial{v^j}}{\partial{v^i}}
\end{vmatrix}
\begin{vmatrix}
\frac{\partial{\omega^i}}{\partial{\dot{q}^i}}\\
\frac{\partial{v^i}}{\partial{\dot{q}^i}}
\end{vmatrix}
\dot{q}^i
\end{equation}

В этом выражении $\omega^i$ и $v^i$ - векторные скоростные параметры $i$-ой ЛСК. Выражение (\ref{matrix_spd_eq}) позволяет зная обобщенные скорости определить скорости всех звеньев цепи. 

До этих пор все вычисления велись в тензорном виде. Разложим вектора по базисам и проанализируем уравнения.
\begin{equation}\label{basis_eq}
V^{(s)}_j = H^s_i\frac{\partial{\dot{P^j}}}{\partial{\dot{P}^i}}^{(i)}\frac{\partial{\dot{P}^i}}{\partial{\dot{q}^i}}^{(i)}\dot{q}^i=H^s_jH^j_i\frac{\partial{\dot{P^j}}}{\partial{\dot{P}^i}}^{(i)}\frac{\partial{\dot{P}^i}}{\partial{\dot{q}^i}}^{(i)}\dot{q}^i
\end{equation}

Выражение 
\begin{equation}\label{}
\frac{\partial{\dot{P}^i}}{\partial{\dot{q}^i}}^{(i)}
=
\begin{vmatrix}
\frac{\partial{\omega^i}}{\partial{\dot{q}^i}}\\
\frac{\partial{v^i}}{\partial{\dot{q}^i}}
\end{vmatrix}^{(i)}=W_i
\end{equation},
разложенное по собственному базису для большого количества реальных кинематических звеньев не зависит от кинематических параметров (хотя в общем случае это не так). Так, например, для поворотного звена этот оператор равен 
\begin{equation}\label{}
W_{rot,i}^{(i)i} = |s[a_x a_y a_z], [0, 0, 0]|^{T(i)}
\end{equation}, где $\bar{a}$ - орт оси вращения в связанной системе координат, а $s$ - маштабный коэффициент. Для линейного кинематического звена оператор имеет вид 
\begin{equation}\label{}
W_{act,i}^{(i)i} = |[0, 0, 0], s[a_x a_y a_z]|^{T(i)}
\end{equation}, где $\bar{a}$ - орт оси трансляции в связанной системе координат, а $s$ - маштабный коэффициент. 
В этих двух наиболее часто встречающихся случаях $W^i$ является константой и может быть определён один раз, на первой итерации алгоритма.

В свою очередь выражение
\begin{equation}\label{}
\frac{\partial{\dot{P^j}}}{\partial{\dot{P}^i}}^{(i)}
=
\begin{vmatrix}
\frac{\partial{\omega^j}}{\partial{\omega^i}} & \frac{\partial{\omega^j}}{\partial{v^i}} \\
\frac{\partial{v^j}}{\partial{\omega^i}} & \frac{\partial{v^j}}{\partial{v^i}}
\end{vmatrix}^{(i)}=R_i^j
\end{equation}
в базисе $i$ принимает вид преобразования переноса пары векторов линейной и угловой скорости к новой точке приложения:
\begin{equation}\label{}
|\omega_j, v_j|^{T(i)} = |\omega_i,\ \omega_i \times \bar{r} + v_i|^{T(i)} = R^{(i)j}_{i}|\omega_i, v_i|^T=
\begin{vmatrix}
E & 0\\
\begin{vmatrix}
0 & r_3 & -r_2\\
-r_3 & 0 & r_1\\
r_2 & -r_1 & 0
\end{vmatrix}
^{(i)} & E
\end{vmatrix}
\begin{vmatrix}
\omega^i\\
v^i
\end{vmatrix}^{(i)}
\end{equation}, где $\bar{r}$ - вектор трансляции, связывающий центры $P_i$ и $P_j$.

C учётом введенных обозначений (\ref{basis_eq}) примет вид:
\begin{equation}\label{maineq}
|\omega_j, v_j|^{T(s)} = H^s_j H^j_i R^j_i W_i \dot{q}^i = H^s_j (H^j_i R^j_i W_i) \dot{q}^i = H^s_j J^{j}_i \dot{q}^i
\end{equation}

В выражении (\ref{maineq}) тензор 
\begin{equation}\label{}
J_i^j = J_i^j(q_{i+1}...q_j)
\end{equation} 

Тензор $J$ является скоростной матрицей Якоби для всех звеньев цепи. Подматрица $J^j_i$ является функцией кинематических параметров, связывающих ЛСК $i$ и ЛСК $j$.

%Запись выражения (\ref{compspd2}) через матрицу Якоби (взятую по скоростям).
Выпишем одyy из cnhjr этой матрицы, соответствующий $n$-ному звену механизма.
\begin{equation}\label{}
J_i^{j=n} =  
\begin{vmatrix}
H^n_1R^n_1W_1 \\
H^n_2R^n_2W_2 \\
... \\
H^n_{n-1}R^n_{n-1}W_{n-1} \\
W_{n} \\
... \\
0
\end{vmatrix}^T
\end{equation}

Таким образом,для нахождения частных производных скорости по скоростям кинематических параметров необходимо для каждой предыдущей координаты взять известную векторную константу $W_i$, преобразовать её оператором приведения $R^n_i$ и перевести в связанный базис (либо сразу в вычислительный базис). 

Алгоритм поиска скоростной матрицы Якоби для $n$-ой СК в $j$-ом базисе может выглядеть следующим образом:

1. На основе текущих координат $q_i$ вычислить трансформации $H^n_i$.
	- либо как композицию обратных преобразований,
	- либо как отношение абсолютных положений в некоторой СК. 
2. Вычислить $R(r, W_i(\bar{q},[1]))$, где r - вектор трансляции $H^n_i$.
3. Перевести полученные 6-вектора в вычислительный базис $j$. Это может быть 
	- собственный базис звена $n = j$
	- лабораторный базис $j = 0$
	- или любой другой базис $j != n,\ j != 0$

\newpage
\section{Общие соображения об управлении сложными системами. Декомпозиция траекторной задачи и задачи стабилизации применительно к многозвенному манипулятору.}

С увеличением сложности объекта управления более успешными становятся сложные, многокаскадные системы управления. Попытка внедрения функционала прохождения траектории или обхода препятствий непосредственно в алгоритм системы стабилизации или системы задания координат, хотя и может претендовать на достижения неких оптимальных показателей, представляется сомнительной с инженерной точки зрения, поскольку приводит к сильной связности алгоритмов управления, неочевидности поведения системы, требует трудоёмкой аналитической переработки при внесении изменений в функционал управления, повышает требования к квалификации работников технического сопровождения.

Разделив задачу управления на несколько слабо связанных задач, можно использовать каждое из них для введения уставки следующей. Так мы можем задавшись некоторыми ограничениями на динамические возможностями нашей системы, построить решение задачи траекторного управления достаточной точности, учитывающее обход препятствий всеми звеньями кинематической цепи без необходимости просчета временных функции кинематических координат $q_i(t)$. 

С другой стороны мы обеспечиваем решение задачи координатного управления, чтобы все уставки $P_i(t)$ выданные для каждого момента времени траекторным алгоритмом выполнялись с необходимой точностью.

Такое разбиение позволяет координатному решателю не заботиться о геометрических ограничениях ввиду того, что они учтены траекторным решателем, а траекторный решатель может расчитывать управление траектории в удобных для этой задачи координатах, поскольку не связан необходимостью физического управления изделием.   

\begin{center}
  \includegraphics[width=\textwidth,height=\textheight,keepaspectratio]{traj.png}
  \captionof{figure}{Взаимное расположение систем координат.}
  \label{}
\end{center}

\newpage
\section{Обратная скоростная задача.}\label{invspd_sect}

Имея решение прямой скоростной задачи построим такое управление $\dot{q}^{*}$, чтобы удовлетв 

\newpage
\section{Следящая система управления положением звена.}

Получив решение обратной скоростной задачи мы получили возможность синтезировать управления заданного 6-скоростью объекта положения звена. Однако, обычно, целью позиционера является управлением положением звена, а не его скоростью.

Прямое решение задачи управления положением через интеграл скоростного управления, очевидно, очень быстро приведет к накоплению вычислительной ошибки.

Чтобы исключить вычислительную ошибку зададимся модельным положением управляемого звена. Пусть задача траекторного управления вырабатывает уставку в виде текущих объектов положения $U(t)$ и скорости $\dot{U}(t)$.

Используем объект уставки как основной сигнал управления (прямое управление), а объект положения для вычисления ошибки $E(t)$ в цепи обратной связи.

Тогда система управления будет отрабатывать управляющее воздействие $\dot{U}(t)$, замешивая в него с небольшим коэффициентом ошибку по положению $E(t)$.

Компенсация $E(t)$ вносится в форме 6-вектора положения взвешенного на соответствующий коэффициент усиления.\ref{geom}. 

Если сумирование компоненты линейной скорости уставки и взвешенного сигнала ошибки положения не вызывает вопросов в силу линейности преобразований координат и линейных скоростей, то сумирование компонент угловой скорости и взвешенных компонент вектора поворота необходимо обосновать.

Возьмём вектор компенсации $\Omega_e$ сонаправленный сигналу ошибки углового положения $\rho_e$ и равный по модулю $K|\rho_e|$, где $K$ размерный коэффициент приведения.
Разложим вектор компенсации $\Omega_e$ на аксиальную и тангенсальную к вектору $\omega_u$ направления.   

Тогда сумарный вектор угловой скорости будет иметь вид 
\begin{equation}
\omega = |\omega_u + \omega_e^tang|\bar{t} + |\omega_e^norm|\bar{n}
\end{equation} 
Тангенсальная компонента $\omega_e^tang$ прямо корректирует модуль $\omega_u$, ускоряя или замедляя воздействие исходя из текущей ошибки. Компонент $\omega_e^norm$ подворачивает звено к ориентации модельного положения. В силу ортогональности управления по тангенсальному и нормальному ортам можно считать независимым при достаточно малом вычислительном шаге.

Надо отметить, что ошибка углового положения более неприятна для системы в целом, поскольку приводит к повороту базиса и переносу ошибки на компоненты линейного положения. Соответственно, следует либо устанавливать высокие требования к динамической точности такой системы, либо осуществлять предобработку сигнала управления путём переноса в связанный базис. 

Анализ устойчивости такой системы управления выходит за рамки настоящего изложения.

Следует также отметить, что может быть построена система управления исключительно по сигналу $E(t)$ без прямого управления по $U(t)$. Такая система имеет лучшую устойчивость и не имеет проблемы переноса ошибки углового положения на компоненты линейного, но имеет худшее быстродействие и может иметь статическую ошибку в режиме движения с ненулевой скоростью.

\newpage
\section{Построение масива траекторных точек для решения задачи линейной интерполяции.}

\newpage
\section{Учет ограничений и весов на уровне координатного решателя. Метод исключения координат.}

При том, что концептуально учет ограничений должен производится на уровне траекторного решателя и приблизительно формулироваться как <<Вышестоящей системе не следует задавать управления, которое нижестоящая не может выполнить>>, полезно рассмотреть возможности внесения ограничений на уровне кординатного решателя.

Как было указано ранее в разделе \ref{invspd_sect}, есть множество вариантов выбора коэффициентов линейной комбинации, удовлетворяющих заданной траектории. Выбор конкретной комбинации зависит от выбранного математического метода и его параметров. Имея на входе 6-вектор потребной скорости, а на выходе вектор производных кинематических координат, координатный решатель с точки зрения остальной системы представляет собой черный ящик и слабо связан с её работой в целом.

Таким образом, изменяя внутреннее устройство координатного решателя мы можем добиться поведения системы, учитывающего необходимые ограничения.
\begin{equation}
\dot{q}(t) = F(V(t), Z(q,\dot{q},t)) 
\end{equation}, где F - функция координатного решателя, а $Z(q,\dot{q},t))$ - дополнительные аргументы, учитывающие текущие ограничения.   

Возможности введения ограничений напрямую зависит от выбраного $F$ метода метода решения поиска комбинации. Некоторые классы рассматриваемых ограничений не могут быть достигнуты при одном выборе функции $F$, но могут быть достигнуты при другом. Исходя из этого, конкретный выбор $F$ должен осуществляться из соображений оптимальности вычислительной реализации в конкретном решателе и возможности учёта необходимых ограничений.

В разделе \ref{invspd_sect} были рассмотрены два метода поиска вектора коэффициентов линейной комбинации. Рассмотрим возможность внесения ограничений в каждый из них.

Метод координатного спуска достаточно легко подвергается модификации и имеет множество параметров. Мы можем варьировать множители для спуска по конкретным координатам вплоть до возможности обнуления приращения в случае, если скалярное произведение целевого 6-вектора и 6-вектора чувствительности по соответствующей координате направлены в неудобную сторону, например если дальнейшее движение по этой координате приведет к столкновению с ограничителем. Метод координатного спуска позволяет достаточно вольно учитывать ограничения и удобен для их внесения. Анализ конкретных алгоритмов введения весов и ограничений выходит за рамки настоящего изложения. 

Метод решения СЛАУ через поиск псевдообратной матрицы гораздо более строг алгоритмически и не позволяет такого вольного обращения с собой, как метод координатного спуска. Однако и здесь мы можем рассмотреть некоторые возможности влиять на результат вычисления.

Один из возможных способов выполнить ограничения на координату
\begin{equation}
q^i_{min} < q^i < q^i_{max} 
\end{equation} является метод исключения координаты. При его использовании, если решение СЛАУ даёт $\dot{q}^i$ такое что, $q^i$ стремится покинуть рабочую зону, мы исключаем координату $q^i$ из вектора кинематических параметров и проводим решение СЛАУ повторно. $\dot{q}^i$ при этом считается равным нулю на текущей итерации. Следует учесть, что такой метод может приводить к резкому останову и не быть приемлемым.

Возможность введения весов....

Автор полагает, что может быть найден метод решения поиска поиска линейной комбинации, лучше подходящий для внесения весов и ограничений, вместе с тем достаточно хороший для вычислительной реализации в ЦПУ или ПЛИС. 

Удобно то, что декомпозиция на траекторную и координатную задачи позволяет эксперементировать с координатным алгоритмом независимо от архитектуры системы в целом.

\newpage
\section{Учет ограничений на уровне траекторного решателя. Разбиение кинематической цепи.}

\newpage
\section{Управление манипулятором в движущемся локальном базисе.}
На основе субъективных ощущений можно сделать вывод, что человек и, интерполируя, прочие живые организмы, имеющие организацию двигательных подсистем мозга близких к нам строят геометрические модели окружающего мира на основе органов чувств с центрами в районе головы (сенсорная модель), и туловища (двигательная модель).

Эти геометрические модели используются для управления конечностями в условиях

\newpage 
\section{Совместное управление частично пересекающимися кинематическими цепями. Древовидная кинематическая цепь.}

Ярким примером задачи частично пересекающейся кинематической цепи является задачу совместного управления положением пальцев руки. Хотя автор и не возьмётся утверждать, что живые организмы комплексно решают задачу управления пальцами конечностей, можно показать, что такое управление возможно и реализуемо в рамках настоящего метода.

\newpage
\section{Выводы.}
Дальнейшее развитие робототехнических систем, в частности антропоморфных роботов, роботов обладающих сложным поведением требует внедрения методов управления движения претендующих на определённую общность, допускающих возможность решения широкого круга задач при учете широкого и, возможно, заранее неизвестного класса ограничений на управляющее воздействие.

Декомпозиция задачи позволяет.

\newpage
В силу линейности координатных преобразований. 
ddP\_s / dd q\_n = ddP\_s' / dd q\_n'

 = sum( dd P\_s` / dd P\_n` * dd P\_n` / dd q\_n` * q\_n' )

 Представим тензор P\_n' в векторной форме как пару тензоров (v,w), определяющих линейную и угловую скорости соответствующей СК. При этом выражение 
 dd P\_s` / dd P\_n` примет форму оператора, вычисляющего состовляющую скоростей (v\_s, w\_s), вызванных (v\_n, w\_n). Обозначим этот оператор как R\_s\_n.

(v\_s, w\_s) = sum(  R\_s\_n * dd (v\_n,w\_n) / dd q\_n' * q\_n' )

Заметим, что в базисе n выражение dd (v\_n,w\_n) / dd q\_n' является константой и зависит от геометрии конкретного кинематического звена. Обозначим величину (dd (v\_n,w\_n) / dd q\_n')(n) как W\_n.

Запишем уравнение (* ) в базисе СК x.
(v\_s, w\_s)(x) = sum( H\_x\_n * R\_s\_n(n) * W\_n * q\_n' )
(v\_s, w\_s)(x) = H\_x\_s(x) * sum( H\_s\_n(s) * R\_s\_n(n) * W\_n * q\_n' )
(v\_s, w\_s)(s) = sum( H\_s\_n(s) * R\_s\_n(qn+1..s)(n) * W\_n * q\_n' )


Здесь H\_x\_n - это оператор гомогенного преобразования базиса n в базис x. Для векторных величин (v,w) такое преобразование означает применение вращения без трансляции.

Выражение (* ) является основным соотношением рассматриваемого метода.

Физический смысл его таков: Чтобы 

Обозначим H(i,j) - линейный оператор преобразования j-ой системы координат в i-ую. 

В текущей работе предлагается метод практической реализации алгоритма управления таким изделием.

Таким образом скорость n-ого звена в декартовой системе координат определяется суммой вкладов обобщенных скоростей кинематических коодинат. Вклад конкретной обобщенной скорости явяляется функцией этой скорости и промежуточных обобщенных координат СК (qn', qn+1 ... qs).

Выведенное соотношение соответствует прямой задаче кинематики системы, записанному в дифференциальной форме.

Выражение (* ) можно переписать в виде:
(v\_s, w\_s)(s) = sum( W\_n(q\_n ... q\_s) * k\_n ). Здесь W\_n - имеет смысл 6-вектора компоненты скорости, а k\_n - некий скалярный множитель. Таким образом (v\_s, w\_s) является линейной комбинацией векторов W\_n(q\_n ... q\_s).

Зададимся целью найти набор коэффициентов, соответствующих заданной 6-скорости (v\_s, w\_s). В общем виде задача может иметь одно решение, множество решений или не иметь решений вообще. Можно, однако утверждать, что, если заданный вектор скорости геометрически осмыслен, то задача будет иметь не менее одного решения.

В матричной форме уравнение (* ) может быть записано как:
(v\_s, w\_s) = A * K = (......)

Это уравнение может быть решено методом поиска псевдообратной матрицы, или каким-либо методом линейного программирования. Поскольку задача имеет множество решений, оптимизируемый функционал может быть введен и использован для выбора конкретного решения.

Построение следящей системы. 

На основе соотношения (* ) может быть построена следящая система положения выходного (или любого другого) звена. Для этого следует ввести положение уставки U. Тогда ошибка E системы от текущего положения P будет иметь вид:
E = U * P**-1. 

Выбрав базис (рекомендуется работа в базисе непосредственно связанном с контролируемым звеном), переведем E в 6-вектор (r, rho), где r - вектор трансляции, а rho - вектор поворота. Используем соотношение (* ) вычислим набор компонент k\_i, такой что (v\_s, w\_s) будет коллинеарным (r, rho). Вектор k\_i определен с точностью до множителя. Длина вектора может быть выбрана из физических характеристик системы и условий наложенных на управление. В простейшем случае длина вектора k\_i может линейно зависеть от длины 6-вектора (r\_s, rho\_s), что будет соответствовать апереодическому процессу регулирования.   

Построение массива точек для прохода по алгоритму линейной интерполяции. 

Соотношение (* ) может быть применено для построения массива векторов q\_i, соответствующих выполнению траектории P\_s(t). Для этого следует дискретизировать траекторию P\_s(t\_i) по массиву 0 < t\_i < t. Дискрет может быть выбран из соображений точности траектории и возможностей вычислителя.

Пусть D - тензор перехода между двумя соседними положениями.
D\_s\_s+1 = P\_s+1(q\_i) * P\_s(q\_j)**-1

Вычислим D как 6-вектор (r, rho). Итеративно вычисляя компоненты k\_i для текущего модельного положения используем виртуальную следящую систему, переместим виртуальную модельную систему из положения P\_s в положение P\_s+1. Геометрически это соответствует численному вычислению минимума функции E(q\_i) = U * P**-1, методом градиентного спуска. Так как кинематические уравнения в общем случае нелинейны, метод может иметь недостатончно хорошую сходимость. Сходимость метода тем лучше, чем плотнее взят набор точек P\_s(t\_i).  



\end{document}

