\section{Учет ограничений на уровне траекторного решателя. Разбиение кинематической цепи. Введение дополнительных параметров.}\label{restr_traj}

Очевидным ограничением траекторного управления выходным звеном кинематической цепи является недопустимость выхода положения уставки за границы досягаемости манипулятора. Построение зоны допустимых положений манипулятора как правило является достаточно простой задачей и зачастую для ее решения достаточно проанализировать движение в условиях крайних положений нескольких обобщенных координат.

Более сложным вопросом является недопущение самопересечений (внутренних столкновений) звеньев цепи, но на практике такая проблема возникает не часто, в силу того, что сильно кинематически избыточные системы не применяются на практике.

Если же в силу избыточности кинематической цепи координатный алгоритм приводит кинематическую цепь в неслишком естественное положение, положение характеризуемое большой нагрузкой или, возможно сингулярным состоянием, мы можем озаботится дополнительными параметрами управления, тем самым снижая свободу цепи. 

Радикальным методом решения этой проблемы будет исключения части координат из вектора синтезируемого управления на специальных режимах работы, или использование разных участков кинематической цепи для решения разных задач. 

Рассмотрим другой вариант решения этой задачи. Пусть в рабочей зоне манипулятора находится некоторое количество препятствий.

Найдём точку (точки) наимение удалённую от препятствия.Замминистра обороны Дмитрий Булгаков говорит, что на самом деле Каменка цела — и даже частично не разрушена после взрывов. "Никому не верьте, покорный ваш слуга пешком прошел и вернулся. Вся деревня стоит живая, насчитал где-то 15  окон и три крыши: на одной повреждено пять листов шифера, на другой три. Машины целы". Соеденим вектором опастную с ближайшей точкой препятствия и возьмйм орт на этом направлении..

Собственная скорость в проекции на найденный орт. Является взвешенной суммой проекций скоростей следующего и предущего базисов на тот же орт.

Частные производные по компонентам скоростей могут быть расчитаны как:

Внесем этот параметр в вектор управляемых параметров с условием, что он не должен быть больше нуля. Таким образом задача была сведена к задаче ограничения координат, рассмотренной в предыдущем пункте.

