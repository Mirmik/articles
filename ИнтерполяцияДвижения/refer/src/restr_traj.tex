\section{Учет ограничений на уровне траекторного решателя. Разбиение кинематической цепи.}\label{restr_traj}

Очевидным ограничением траекторного управления выходным звеном кинематической цепи является недопустимость выхода положения уставки за границы досягаемости манипулятора. Построение зоны допустимых положений манипулятора как правило является достаточно простой задачей и зачастую для ее решения достаточно проанализировать движение в условиях крайних положений нескольких обобщенных координат.

Более сложным вопросом является недопущение самопересечений (внутренних столкновений) звеньев цепи, но на практике такая проблема возникает не часто, в силу того, что сильно кинематически избыточные системы не применяются на практике.

Если же в силу избыточности кинематической цепи координатный алгоритм приводит кинематическую цепь в неслишком естественное положение, положение характеризуемое большой нагрузкой или, возможно сингулярным состоянием, мы можем озаботится дополнительными параметрами управления, тем самым снижая свободу цепи. Радикальным методом решения этой проблемы будет исключения части координат из вектора синтезируемого управления на специальных режимах работы, или использование разных участков кинематической цепи для решения разных задач. 

Рассмотрим другой вариант решения этой задачи
