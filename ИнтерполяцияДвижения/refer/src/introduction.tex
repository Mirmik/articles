\section{Введение.}

В настоящее время бурное развитие получили робототехнические системы, манипуляторы, позиционеры, станки ЧПУ, выполняющие задачу изменения положения рабочего органа (инструмента, схвата, зонда) в лабораторной системе координат и характеризыющиеся цепной компоновкой кинематических звеньев.

Обратная задача кинематики, решение которой требуется для управления такими изделиями нелинейна, может иметь множество решений и представлять значительную вычислительную сложность. В связи с этим было разработано множество методов решения обратной задачи кинематики.

Часть методов, в особенности аналитических, построена для решения частных случаев задачи управления кинематической цепью и может быть применена только к ограниченному числу изделий. К числу этой категории методов также относятся методы, работающие только с одним типом кинематических звеньев (в основном с одностепенными поворотными звеньями).

Наиболее общее, не зависящее от геометрии решение даёт применение матрицы Якоби, описывающей связь положения выходного звена с кинематическими координатами в частных производных.

Само по себе использование матрицы Якоби для расчета управления звеном кинематической цепи может быть не слишком удобным из-за неинтуитивности частных производных компонент тензора положения. В значительной степени неудобство управления в частных производных координат положения происходит из неочевидности физического смысла производных компонент объекта ориентации, который может задаваться матрицей размерности 3x3, или кватернионом, или быть частью более сложного алгебраического объекта, определяющего положение звена целиком, а именно матрицы размерности 4x4 или бикватерниона. Это ухудшает прозрачность алгоритма и увеличивает требования к квалификации разработчика програмного обеспечения конкретного изделия.

Вместе с тем, построение сложных алгоритмов управления в реальном времени требует предсказуемости поведения позиционера в общем случае движения, в то время как частные производные компонент кватерниона углового положения или матрицы поворота, могут иметь довольно странное поведение при совместном многокоординатном управлении в силу нелинейности компонент, при использовании их в качестве параметров системы регулирования. 

Для настоящего метода выбрано решение через скоростной аналог матрицы Якоби, характеризующий связь производных кинематических параметров с векторами линейной и угловой скоростей звеньев манипулятора. В работе показано, как решение скоростной задачи может быть использовано для координатно-траекторного управления широким классом открытых кинематических цепей. На основе метода приводится вариант алгоритма, достаточный для построения на его основе следящих систем, систем линейной интерполяции, а также достаточно простой для реализации в ЦПУ, ПЛИС, для управления реальным физическим объектом или моделью, а также обсуждаются варианты его модификации под ограничения конкретного объекта.

Следует отметить, что метод не налагает ограничения на класс кинематических пар и, хотя и разрабатывался для кинематических пар пятого класса, может быть применен для управления любой открытой кинематической цепью как из простых, так и из сложных звеньев.

В настоящей работе вводится и используется математическая нотация, совмещающая в рамках одного алгебраического объекта такие понятия как линейное и угловое положения, линейная и угловая скорость и так далее. Такой выбор математической нотации обусловлен практической направленностью метода. Использование указанных пар, как обособленных вычислительных объектов позволяет упростить сопровождение програмного обеспечения и уменьшить вероятность допущения ошибок на этапе переноса математики в исходный код программы, уменьшить требования к математической подготовке разработчиков прикладного програмного обеспечения. Вместе с тем автор полагает, что подобная нотация имеет ценность сама по себе и дальнейшее исследование в этом направлении может позволить строить системы стабилизации для широкого класса объектов управления, имеющих хорошую угловую динамику и осуществляющих работу в трёхмерном пространстве, таких как космические и подводные аппараты, мультикоптеры и прочее.

В применении к задаче управления позиционером/манипулятором данная нотация позволяет существенно упростить запись уравнений, их анализ, а также позволяет изложить задачу в более интуитивно понятной форме.

В последних частях работы обсуждается возможность применения метода к управлению манипуляторами автономных роботов, древовидными кинематическими цепями, многопальцевыми схватами. 