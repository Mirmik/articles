\section{Учет ограничений и весов на уровне координатного решателя. Метод исключения координат.}

При том, что концептуально учет ограничений должен производится на уровне траекторного решателя и приблизительно формулироваться как <<Вышестоящей системе не следует задавать управления, которое нижестоящая не может выполнить>>, полезно рассмотреть возможности внесения ограничений на уровне кординатного решателя.

Как было указано ранее в разделе \ref{invspd_sect}, есть множество вариантов выбора коэффициентов линейной комбинации, удовлетворяющих заданной траектории. Выбор конкретной комбинации зависит от выбранного математического метода и его параметров. Имея на входе 6-вектор потребной скорости, а на выходе вектор производных кинематических координат, координатный решатель с точки зрения остальной системы представляет собой черный ящик и слабо связан с её работой в целом.

Таким образом, изменяя внутреннее устройство координатного решателя мы можем добиться поведения системы, учитывающего необходимые ограничения.
\begin{equation}
\dot{q}(t) = F(V(t), Z(q,\dot{q},t)) 
\end{equation}, где F - функция координатного решателя, а $Z(q,\dot{q},t))$ - дополнительные аргументы, учитывающие текущие ограничения.   

Возможности введения ограничений напрямую зависит от выбраного $F$ метода метода решения поиска комбинации. Некоторые классы рассматриваемых ограничений не могут быть достигнуты при одном выборе функции $F$, но могут быть достигнуты при другом. Исходя из этого, конкретный выбор $F$ должен осуществляться из соображений оптимальности вычислительной реализации в конкретном решателе и возможности учёта необходимых ограничений.

В разделе \ref{invspd_sect} были рассмотрены два метода поиска вектора коэффициентов линейной комбинации. Рассмотрим возможность внесения ограничений в каждый из них.

Метод координатного спуска достаточно легко подвергается модификации и имеет множество параметров. Мы можем варьировать множители для спуска по конкретным координатам вплоть до возможности обнуления приращения в случае, если скалярное произведение целевого 6-вектора и 6-вектора чувствительности по соответствующей координате направлены в неудобную сторону, например если дальнейшее движение по этой координате приведет к столкновению с ограничителем. Метод координатного спуска позволяет достаточно вольно учитывать ограничения и удобен для их внесения. Анализ конкретных алгоритмов введения весов и ограничений выходит за рамки настоящего изложения. 

Метод решения СЛАУ через поиск псевдообратной матрицы гораздо более строг алгоритмически и не позволяет такого вольного обращения с собой, как метод координатного спуска. Однако и здесь мы можем рассмотреть некоторые возможности влиять на результат вычисления.

Один из возможных способов выполнить ограничения на координату
\begin{equation}
q^i_{min} < q^i < q^i_{max} 
\end{equation} является метод исключения координаты. При его использовании, если решение СЛАУ даёт $\dot{q}^i$ такое что, $q^i$ стремится покинуть рабочую зону, мы исключаем координату $q^i$ из вектора кинематических параметров и проводим решение СЛАУ повторно. $\dot{q}^i$ при этом считается равным нулю на текущей итерации. Следует учесть, что такой метод может приводить к резкому останову и не быть приемлемым.

Автор полагает, что может быть найден метод решения поиска поиска линейной комбинации, лучше подходящий для внесения весов и ограничений, вместе с тем достаточно хороший для вычислительной реализации в ЦПУ или ПЛИС.

Особый интерес представляет поиск такого алгоритма, в котором положительная и отрицательная скорости по каким-либо координатам будут иметь разные веса. 

Удобно то, что декомпозиция на траекторную и координатную задачи позволяет эксперементировать с координатным алгоритмом независимо от архитектуры системы в целом.