\section{Следящая система управления положением звена.}

Получив решение обратной скоростной задачи мы получили возможность синтезировать управления заданного 6-скоростью объекта положения звена. Однако, обычно, целью позиционера является управлением положением звена, а не его скоростью.

Прямое решение задачи управления положением через интеграл скоростного управления, очевидно, очень быстро приведет к накоплению вычислительной ошибки.

Чтобы исключить вычислительную ошибку зададимся модельным положением управляемого звена. Пусть задача траекторного управления вырабатывает уставку в виде текущих объектов положения $U(t)$ и скорости $\dot{U}(t)$.

Используем объект уставки как основной сигнал управления (прямое управление), а объект положения для вычисления ошибки $E(t)$ в цепи обратной связи.

Тогда система управления будет отрабатывать управляющее воздействие $\dot{U}(t)$, замешивая в него с небольшим коэффициентом ошибку по положению $E(t)$.

Компенсация $E(t)$ вносится в форме 6-вектора положения взвешенного на соответствующий коэффициент усиления.\ref{geom}. 

Если сумирование компоненты линейной скорости уставки и взвешенного сигнала ошибки положения не вызывает вопросов в силу линейности преобразований координат и линейных скоростей, то сумирование компонент угловой скорости и взвешенных компонент вектора поворота необходимо обосновать.

Возьмём вектор компенсации $\Omega_e$ сонаправленный сигналу ошибки углового положения $\rho_e$ и равный по модулю $K|\rho_e|$, где $K$ размерный коэффициент приведения.
Разложим вектор компенсации $\Omega_e$ на аксиальную и тангенсальную к вектору $\omega_u$ направления.   

Тогда сумарный вектор угловой скорости будет иметь вид 
\begin{equation}
\omega = |\omega_u + \omega_e^tang|\bar{t} + |\omega_e^norm|\bar{n}
\end{equation} 
Тангенсальная компонента $\omega_e^tang$ прямо корректирует модуль $\omega_u$, ускоряя или замедляя воздействие исходя из текущей ошибки. Компонент $\omega_e^norm$ подворачивает звено к ориентации модельного положения. В силу ортогональности управления по тангенсальному и нормальному ортам можно считать независимым при достаточно малом вычислительном шаге.

Надо отметить, что ошибка углового положения более неприятна для системы в целом, поскольку приводит к повороту базиса и переносу ошибки на компоненты линейного положения. Соответственно, следует либо устанавливать высокие требования к динамической точности такой системы, либо осуществлять предобработку сигнала управления путём переноса в связанный базис. 

Анализ устойчивости такой системы управления выходит за рамки настоящего изложения.

Следует также отметить, что может быть построена система управления исключительно по сигналу $E(t)$ без прямого управления по $U(t)$. Такая система имеет лучшую устойчивость и не имеет проблемы переноса ошибки углового положения на компоненты линейного, но имеет худшее быстродействие и может иметь статическую ошибку в режиме движения с ненулевой скоростью.