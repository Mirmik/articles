\section{Управление манипулятором в движущемся локальном базисе.}

На основе субъективных ощущений можно сделать вывод, что человек и, интерполируя, прочие живые организмы, имеющие организацию двигательных подсистем мозга близких к нам строят геометрические модели окружающего мира на основе органов чувств с центрами в районе головы (сенсорная модель), и туловища (базовая модель).

Эти геометрические модели используются для управления конечностями в условиях, когда тело перемещается относительно объектов окружающего мира.

Построение автономных роботов (не привязанных к лабораторной системе координат) затруднительно без использования связанных вычислительных базисов.

Автономный робот должен решать задачи управления манипуляторами в условиях, когда его тело перемещается в пространстве, а условная голова, на которой закреплены сенсоры, перемещается относительно условного туловища. 

Рассмотрим, как с помощью настоящего метода можно решить задачу управления схватом манипулятора когда база (условное туловище) имеет ненулевую 6-скорость относительно захватываемого объекта, а сенсор (условная голова) имеет ненулевую 6-скорость относительно базы. Предположим пока, что 6-вектора ускорений базы, захватываемого объекта относительно, и сенсора. 

Очевидно, что получение координат захватываемого тела возможно только в рамках сенсорной модели. Назовем $S$ объект преобразования положений в сенсорной модели относительно базовой модели.

В базисе связанной модели положение объекта реального мира может быть записано как
\begin{equation}
P_o^{(b)} = S^{-1} * P_o^{(s)}
\end{equation}
Его же 6-скорость будет иметь более сложный вид.
\begin{equation}\label{bura}
V_o^{(s)} = V_o^{(s)} - \dot{S}^{(s)} \times P_o^{(s)}
\end{equation}
\begin{equation}
V_o^{(b)} = S^{(b)-1} V_o^{(s)} - V_b^{(b)} = S^{(b)-1} (V_o^{(s)} - \dot{S}^{(s)} \times P_o^{(s)}) - V_b^{(b)}
\end{equation}

Запись \ref{bura} соответствует обобщению формулы Бура для объектов гомогенных преобразований. Оператор $\times$ здесь эквивалентен ранее расмотренным в разделе $\ref{straight}$ оператору $R^j_i$.

Используя приведенные выше соотношения можно вычислить положение объекта и его 6-скорость в базисе $(b)$. 

Теперь задача управления манипулятором может быть сведена к задачи перехвата объекта движущегося с постоянной 6-скоростью в лабораторной системе координат. 

Предположим для простоты изложения, что для захвата объекта достаточно совместить объект положения схвата с объектом положения захватываемого тела в какой-то момент времени.

Это задача траекторного управления может быть решена множеством разных способов. Решение задачи перехвата зависит от динамических ограничений на управление динамикой звеньев. Для выбора точки перехвата можно разбить предсказываемую траекторию захватываемого тела в базисе $(b)$ и найти положение наименее движение к которому будет минимизировать заданный функционал. Если улговаяая динамика выходного звена достаточно хорошая, можно не взирая на углы осуществить движение по вектору перпендикулярному к траектории объекта, это будет соответствовать минимуму линейной скорости.

Построение системы конкретных алгоритмов перехвата, анализ задачи в условиях ненулевых 6-ускорений, а также анализ возможности задания управления положением базы, по дополнительному набору обобщенных координат, влияющих в том числе на положение вычислительного базиса в терминах объектов гомогенных преобразований может быть практически ценной задачей и интересным направлением исследования.