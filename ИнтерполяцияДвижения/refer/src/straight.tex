\section{Прямая скоростная задача.}\label{straight}

Рассмотрим производную тензора положения $j$-ой СК в цепочке, состоящей из $n$ звеньев.
\begin{equation}\label{eq1}
V^j = \frac{dP_j}{dt} = \sum_{i=1}^{n}\frac{\partial{P_j}}{\partial{q_i}}\dot{q_i} 
\end{equation}

Поскольку от $q_i$ напрямую зависит только $P_i$, а вариации остальных $P_j, j \neq i$ являются зывисимыми,
\begin{equation}\label{}
\frac{\partial{P_j}}{\partial{q_i}} = \frac{\partial{P_j}}{\partial{P_i}}\frac{\partial{P_i}}{\partial{q_i}}
\end{equation}

Причем, очевидно, что 
\begin{equation}\label{}
\frac{\partial{P_j}}{\partial{P_i}} = 0, \ \ \forall i: i > j
\end{equation}
, поскольку эволюции последующих звеньев не могут влиять на положение предшествующих.

Возпользуемся независимостью вариаций для перехода к скоростным параметрам:
\begin{equation}\label{}
\frac{\partial{P_j}}{\partial{P_i}} = \frac{\partial{P_j}\partial{t}}{\partial{P_i}\partial{t}} = \frac{\partial{\dot{P}_j}}{\partial{\dot{P}_i}}
\end{equation}
\begin{equation}\label{}
\frac{\partial{P_i}}{\partial{q_i}} = \frac{\partial{P_i}\partial{t}}{\partial{q_i}\partial{t}} = \frac{\partial{\dot{P}_i}}{\partial{\dot{q}_i}}
\end{equation}

Тогда (\ref{eq1}) можно записать в тензорном виде как:
\begin{equation}\label{eq2}
V^j = \frac{\partial{\dot{P^j}}}{\partial{\dot{P}^i}}\frac{\partial{\dot{P}^i}}{\partial{\dot{q}^i}}\dot{q}^i 
\end{equation}

Переходя к записи в интересующих нас компонентах скоростей имеем:
\begin{equation}\label{matrix_spd_eq}
\begin{vmatrix}
\omega^j\\
v^j
\end{vmatrix}
=
\begin{vmatrix}
\frac{\partial{\omega^j}}{\partial{\omega^i}} & \frac{\partial{\omega^j}}{\partial{v^i}} \\
\frac{\partial{v^j}}{\partial{\omega^i}} & \frac{\partial{v^j}}{\partial{v^i}}
\end{vmatrix}
\begin{vmatrix}
\frac{\partial{\omega^i}}{\partial{\dot{q}^i}}\\
\frac{\partial{v^i}}{\partial{\dot{q}^i}}
\end{vmatrix}
\dot{q}^i
\end{equation}

В этом выражении $\omega^i$ и $v^i$ - векторные скоростные параметры $i$-ой ЛСК. Выражение (\ref{matrix_spd_eq}) позволяет зная обобщенные скорости определить скорости всех звеньев цепи. 

До этих пор все вычисления велись в тензорном виде. Разложим вектора по базисам и проанализируем уравнения.
\begin{equation}\label{basis_eq}
V^{(s)}_j = H^s_i\frac{\partial{\dot{P^j}}}{\partial{\dot{P}^i}}^{(i)}\frac{\partial{\dot{P}^i}}{\partial{\dot{q}^i}}^{(i)}\dot{q}^i=H^s_jH^j_i\frac{\partial{\dot{P^j}}}{\partial{\dot{P}^i}}^{(i)}\frac{\partial{\dot{P}^i}}{\partial{\dot{q}^i}}^{(i)}\dot{q}^i
\end{equation}

Выражение 
\begin{equation}\label{}
\frac{\partial{\dot{P}^i}}{\partial{\dot{q}^i}}^{(i)}
=
\begin{vmatrix}
\frac{\partial{\omega^i}}{\partial{\dot{q}^i}}\\
\frac{\partial{v^i}}{\partial{\dot{q}^i}}
\end{vmatrix}^{(i)}=W_i
\end{equation},
разложенное по собственному базису для большого количества реальных кинематических звеньев не зависит от обобщенных координат (хотя в общем случае это не так). Так, например, для поворотного звена этот оператор равен 
\begin{equation}\label{}
W_{rot,i}^{(i)i} = |s[a_x a_y a_z], [0, 0, 0]|^{T(i)}
\end{equation}, где $\bar{a}$ - орт оси вращения в связанной системе координат, а $s$ - скалярный коэффициент. Для линейного кинематического звена оператор имеет вид 
\begin{equation}\label{}
W_{act,i}^{(i)i} = |[0, 0, 0], s[a_x a_y a_z]|^{T(i)}
\end{equation}, где $\bar{a}$ - орт оси трансляции в связанной системе координат, а $s$ - скалярный коэффициент. 
В этих двух наиболее часто встречающихся случаях $W^i$ является константой и может быть определён один раз, на первой итерации алгоритма.

В свою очередь выражение
\begin{equation}\label{}
\frac{\partial{\dot{P^j}}}{\partial{\dot{P}^i}}^{(i)}
=
\begin{vmatrix}
\frac{\partial{\omega^j}}{\partial{\omega^i}} & \frac{\partial{\omega^j}}{\partial{v^i}} \\
\frac{\partial{v^j}}{\partial{\omega^i}} & \frac{\partial{v^j}}{\partial{v^i}}
\end{vmatrix}^{(i)}=R_i^{(i)j}(q_{i+1},...,q_j)
\end{equation}
в базисе $i$ принимает вид преобразования переноса пары векторов линейной и угловой скорости к новой точке приложения:
\begin{equation}\label{}
|\omega_j, v_j|^{T(i)} = |\omega_i,\ \omega_i \times \bar{r} + v_i|^{T(i)} = R^{(i)j}_{i}|\omega_i, v_i|^T=
\begin{vmatrix}
E & 0\\
\begin{vmatrix}
0 & r_3 & -r_2\\
-r_3 & 0 & r_1\\
r_2 & -r_1 & 0
\end{vmatrix}
^{(i)} & E
\end{vmatrix}
\begin{vmatrix}
\omega^i\\
v^i
\end{vmatrix}^{(i)}
\end{equation}, где $\bar{r}$ - вектор трансляции, связывающий центры $P_i$ и $P_j$.

C учётом введенных обозначений (\ref{basis_eq}) примет вид:
\begin{equation}\label{maineq}
|\omega_j, v_j|^{T(s)} = H^s_j H^j_i R^j_i W_i \dot{q}^i = H^s_j (H^j_i R^j_i W_i) \dot{q}^i = H^s_j J^{j}_i \dot{q}^i
\end{equation}

Тензор $J$ является скоростной матрицей Якоби для всех звеньев цепи. Подматрица $J^j_i$ является функцией кинематических параметров, связывающих ЛСК $i$ и ЛСК $j$. Уплощенная матрица $\textbf{J}$, соответствующая тензору $J^j_i$ имеет размерность $6*N \times N$, где N - количество обобщенных координат цепи. Кроме того оно является верхнетреугольной по подматрицам размерности $6\times1$ То есть, например для 6-звенного позиционера - это матрица размерности $36\times6$, причем $126$ действительных чисел, входящих в нее могут не быть нулевыми и зависят о текущих координат.

Для решения практических задач нет смысла вычислять эту матрицу целиком. Интересны лишь строки, соответствующие параметрам, по которым мы хотим задавать управление. Для управления положением выходного звена - это последние 6 строк матрицы $\textbf{J}$. Если нас интересует только линейное положение манипулятора, или только угловое положение, количество строк сокращается до трёх.

%Запись выражения (\ref{compspd2}) через матрицу Якоби (взятую по скоростям).
Выпишем шестерку строк этой матрицы, соответствующий $n$-ному звену механизма.
\begin{equation}\label{}
\textbf{J}^n =  
\begin{vmatrix}
H^n_1R^n_1W_1 \\
H^n_2R^n_2W_2 \\
... \\
H^n_{n-1}R^n_{n-1}W_{n-1} \\
W_{n} \\
... \\
0
\end{vmatrix}^T
\end{equation} ,где
\begin{equation}
H^i_iR^n_iW_i =
\begin{vmatrix}
\frac{\partial{\omega^n_1}}{\partial{\dot{q}^i}}\\
\frac{\partial{\omega^n_2}}{\partial{\dot{q}^i}}\\
\frac{\partial{\omega^n_3}}{\partial{\dot{q}^i}}\\
\frac{\partial{v^n_1}}{\partial{\dot{q}^i}}\\
\frac{\partial{v^n_2}}{\partial{\dot{q}^i}}\\
\frac{\partial{v^n_3}}{\partial{\dot{q}^i}}
\end{vmatrix}
\end{equation}


Таким образом,для нахождения частных производных скорости по скоростям кинематических параметров необходимо для каждой предыдущей координаты взять известную векторную константу $W_i$, преобразовать её оператором приведения $R^n_i$ и перевести в связанный базис (либо сразу в вычислительный базис). 

Алгоритм поиска скоростной матрицы Якоби для $n$-ой СК в $j$-ом базисе может выглядеть следующим образом:

\begin{enumerate}
\itemНа основе текущих координат $q_i$ вычислить трансформации $H^n_i$.
	- либо как композицию обратных преобразований,
	- либо как отношение абсолютных положений в некоторой СК. 
\itemВычислить $R(r, W_i(\bar{q},[1]))$, где r - вектор трансляции $H^n_i$.
\itemПеревести полученные 6-вектора в вычислительный базис $j$. Это может быть\\ 
	- собственный базис звена $n = j$ \\
	- лабораторный базис $j = 0$ \\
	- или любой другой базис $j != n,\ j != 0$
\end{enumerate}