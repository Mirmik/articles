\section{Выводы и дальнейшие направления исследований.}

В настоящей работе представлен рассмотрен набор методов построения систем следящего управления многозвеньевыми кинематическими цепями. С целью демонстрации общности метода рассмотрены возможности применения метода к различным кинематическим схемам. Созданны предпосылки к построению матаппарата, совмещающего в едином математическом объекте линейные и угловые параметры с целью дальнейшего его применения к построению систем управления положением кинематических цепей, решения задач навигации, ориентациии и стабилизации летательных, подводных и космических аппаратов.

Практически ценными направлениями дальнейших исследований являются:

1. Совмещение траекторно координатной задачи кинематической цепи с управлением силами и моментами для решения задачи схвата в рамках единого алгоритма. 

2. Применения алгоритма в условиях локального вычислительного базиса для управления манипуляторами автономных роботов.

3. Общий анализ динамики систем управления объектами с шестью степенями свободы, совершающих сложное движение в трехмерном пространстве, с применением аппарата гомогенных преобразований.
