\section{Список использованной литературы}

\begin{itemize}
%\begin{description}
\item Salychev O.S. <<Applied Inertial Navigation: Problems and Solutions>>, BMSTU Press, 2004 (ISBN 5-7038-2395-1)\itemПритыкин Д., cерия статей <<Магия тензорной алгебры>>, 2015 \newline href:https://habr.com/ru/post/261421/
\itemКаратаев Е.А., <<Преобразования гиперкомплексных чисел>>, Москва, СОЛОМОН-Пресс, 2017 (ISBN 978-5-91359-249-1)
\itemН.Н. Голованов, <<Геометрическое моделирование>>, Москва, Физматлит, 2002 (ISBN 5-94052-048-0)
\itemКотельников А. П. Винтовое счисление и некоторые приложения его к геометрии и механике. Казань, 1895
\itemЕ.И. Ломовцева, Ю.Н. Челноков, <<Решение обратной задачи кинематики Стэндфордского манипулятора сприменением бикватернионной теории кинематического управления>>.
\itemИ.К. Хапкина, <<Синтез управления роботами с использованием вектора скорости>>.
\itemА.И. Жильцов, В.С. Жуков, Д.А. Рылеев, <<Управление манипуляторами  с числом степеней свободы более шести >> НИИ ИСУ МГТУ им. Н.Э. Баумана, Москва, УДК 681.58.
\itemПоезжаева Е. В., Закиров Е. А., Малёв М. В. <<Кинематика избыточного манипулятора робота для тушения пожаров>>. Молодой ученый. — 2015. — №23. — С. 204-206.
\end{itemize}
%\end{description}