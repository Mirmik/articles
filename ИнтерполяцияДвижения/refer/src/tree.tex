\section{Совместное управление частично пересекающимися кинематическими цепями. Древовидная кинематическая цепь.}

Ярким примером задачи частично пересекающейся кинематической цепи является задачу совместного управления положением пальцев руки. Хотя автор и не возьмётся утверждать, что живые организмы комплексно решают задачу управления пальцами конечностей, можно показать, что такое управление возможно и реализуемо в рамках настоящего метода.