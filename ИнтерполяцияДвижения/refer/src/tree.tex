\section{Совместное управление частично пересекающимися кинематическими цепями. Древовидная кинематическая цепь.}

Древовидная кинематическая цепь является сложной кинематической цепью. Ее матрица частных производных скоростей $J$ может быть построена на основе анализа простых подцепей, входящих в нее. Древовидная цепь (как и было указано в разделе \ref{other}) не имеет верхнетреугольного вида, как простая, но в целом процесс построения матрицы $J$ для нее не сильно отличается.

Ярким примером задачи управления древовидной кинематической цепи является задача совместного управления положением пальцев руки. Хотя автор и не возьмётся утверждать, что живые организмы комплексно решают задачу управления пальцами конечностей, можно показать, что такое управление возможно и реализуемо в рамках настоящего метода.

Задача управления древовидной цепью может быть решена через разбиение сложной цепи на несколько иерархически связанных простых подцепей, управление по которым может осуществляться независимо, но совместное управление позволяет добиться лучшего качества регулирования.

Для построения такого решения построим матрицу $J$ и выберем в качестве управляемых компонент компоненты скоростей выходных звеньев древовидной кинематической цепи. Следует оговориться, что для задачи управления кистью в достаточной степени бессмысленно пытаться управлять ориентацией каждого пальца. Вместо этого можно взять в качестве управляемых компонент линейные компоненты скоростей кончиков захватов и угловую ориентацию кисти. 

Отдельно проанализируем простые кинематические цепи, соединяющие корень дерева с каждым из выходных звеньев цепочки. Теперь, в соответствии с методикой изложенной в \ref{invspd_sect} мы можем построить решение прямой скоростной задачи для каждой из простых цепей. Оформим это решение в виде матрицы \textbf{$J^i$}.

Объединим матрицы дополняя нулевыми элементами в случае, если решение по прямой цепи не содержит компоненты, относящейся к соответствующему звену:

\begin{equation}
\textbf{J} = 
\begin{vmatrix}

\frac{\partial{V}}{\partial{\dot{q}_{v1}}} & 0 & 0 & 0 \\
\frac{\partial{V}}{\partial{\dot{q}_{v1}}} & 0 & 0 & 0 \\
0 & \frac{\partial{V}}{\partial{\dot{q}_{v1}}} & 0 & 0 \\
0 & \frac{\partial{V}}{\partial{\dot{q}_{v1}}} & 0 & 0 \\
0 & 0 & \frac{\partial{V}}{\partial{\dot{q}_{v1}}} & 0 \\
0 & 0 & \frac{\partial{V}}{\partial{\dot{q}_{v1}}} & 0 \\
0 & 0 & 0 & \frac{\partial{V}}{\partial{\dot{q}_{v1}}} \\
0 & 0 & 0 & \frac{\partial{V}}{\partial{\dot{q}_{v1}}} \\
\frac{\partial{V}}{\partial{\dot{q}_{v1}}} &
\frac{\partial{V}}{\partial{\dot{q}_{v1}}} &
\frac{\partial{V}}{\partial{\dot{q}_{v1}}} &
\frac{\partial{V}}{\partial{\dot{q}_{v1}}} \\

\frac{\partial{V}}{\partial{\dot{q}_{v1}}} &
\frac{\partial{V}}{\partial{\dot{q}_{v1}}} &
\frac{\partial{V}}{\partial{\dot{q}_{v1}}} &
\frac{\partial{V}}{\partial{\dot{q}_{v1}}} \\

\frac{\partial{V}}{\partial{\dot{q}_{v1}}} &
\frac{\partial{V}}{\partial{\dot{q}_{v1}}} &
\frac{\partial{V}}{\partial{\dot{q}_{v1}}} &
\frac{\partial{V}}{\partial{\dot{q}_{v1}}} \\

\end{vmatrix}
\end{equation}

Теперь, выбирая из этой матрицы столбцы, отвечающие компонентам требуемого управления, мы можем применить ранее описанный метод для построения следящей системы. 