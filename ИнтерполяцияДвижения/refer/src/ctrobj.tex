\section{Объект управления}

Основными объектами управления рассматриваемого в данной работе метода являются механизмы, роботы, манипуляторы, кинематика которых описываются простыми кинематическими цепями.

Типичным представителем таких механизмов являются роботы-манипуляторы.

... Картинка.

... картинка.

Управление положением схвата первично для таких изделий. Зачастую роботы манипуляторы имеют избыточную кинематику, что ставит вопрос о поиске управления, исключающего коллизии и сунгулярные состояния.

Другой тип рассматриваемых объектов управления, позиционеры, чпу станки:

...

...

Характеризуются меньшей избыточностью кинематики, имеют высокую жесткость и точность позиционирования.

В качестве перспективного направления развития метода рассматривается управление манипуляторами мобильных платформ (колёсных, шагающих).

В качестве перспективного направления развития метода также рассматривается управление сложными схватами.

Эти объекты требуют синхронного многокоординатного управления положением, которое в рамках рассматриваемого метода осуществляется через построение следящей системы с модельной уставкой положения выходного звена.

