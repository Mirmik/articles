\section{Объект управления.}

Рассмотрим открытую кинематическую цепь состоящей из простых кинематических звеньев. С каждым кинематическим звеном свяжем две локальные системы координат, распожив их в точках сочленения кинематических пар. Таким образом каждое кинематическое звено имеет входную и выходную ЛСК, жестко связанные с ним.

В сочленении кинематической пары также оказывается две системы координат - по одной для каждого звена пары (выходная первого звена и входная второго). Введем кинематические параметры звеньев $q_i$, определяющих взаимное расположение локальных систем координат.

\begin{center}
  \includegraphics[width=0.6\textwidth,height=0.6\textheight,keepaspectratio]{ax3view.jpg}
  \captionof{figure}{Взаимное расположение систем координат.}
  \label{}
\end{center}

Таким образом кинематическая цепь определяется цепочкой локальных систем координат, положение каждой из которых определяется положением предыдущей и объектом гомогенного преобразования, либо константным, либо завищим от соответствующего кинематического параметра.

\colorbox{shadecolor}
{\parbox{0.9\textwidth}{Здесь и далее под объектом гомогенного преобразования (англ. homogenous transform) понимается геометрическая сущность, описывающая однородное координатное преобразование, состоящее из комбинации поворота и переноса. В разных частях настоящего изложения объект гомогенного преобразования фигурирует как оператор преобразования СК, способ описания положения кинематического звена имеющий смысл соответствующего тензора, или програмный объект (в терминах ООП), непосредственно используемый в програмном обеспечении. Подробнее в разделе \ref{geom}}}

Пронумеруем звенья и системы координат. Пусть неподвижное кинематическое звено, жестко связанное с лабораторной системой координат считается 0-ым, лабораторная система координат считается 0-ой СК, пусть входная система каждого $n$-ого звена считается $n$-ой СК. Выходную систему кинематического звена обозначим как $n^*$. 

Если в нашей цепи $m$ звеньев, то СК выходного звена цепи будет $m$-ой системой координат. (рассматривать $m^*$ для выходного звена цепи не имеет особого смысла).

Конкретный тип кинематического звена не имеет существенного значения, но желательно, чтобы тензор гомогенного преобразования, связывающий