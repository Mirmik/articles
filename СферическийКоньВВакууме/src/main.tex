\documentclass[a4paper]{article}
\usepackage[12pt]{extsizes}
\usepackage[left=30mm, top=25mm, right=20mm, bottom=25mm, nohead]{geometry}
\usepackage{xcolor}
\usepackage{amssymb}
\usepackage{amsmath}
\usepackage{lipsum}
\setlength{\parindent}{3ex}
\setlength{\parskip}{1em}
\usepackage[utf8]{inputenc}
\usepackage[T2A]{fontenc}
\usepackage[english, russian]{babel}
\usepackage{caption}
\usepackage{graphicx} 
\graphicspath{ {./images/} }
\definecolor{shadecolor}{RGB}{230,230,230}
\begin{document}

%\section{Система управления идеальным изотропным объектом управления в трёхмерном пространстве.}

\section{Возможность обобщения теории автоматического управления к решению задач управления динамическими системами в изотропных 3-мерном пространстве.}
Классический подход теории управления при решении задачи управления динамическими процессами в трёхмерном пространстве, такими как движение летательных, космических аппаратов, позиционеров, манипуляторов и т.д., состоит в разбиении управляющего воздействия по набору управляющих каналов, и независимому решению задачи стабилизации в каждом отдельно взятом канале. Этот подход замечательно работает вблизи выбранных опорных режимов, вблизи которых проведена линеаризация уравнений движения объекта, но его применимость для решения задачи управления эволюциями объекта управления в общем случае вызывает вопросы.

На примере идеальной модели объекта управления рассмотрим возможность построения системы управления, используя в качестве сигналов данные о положении, скоростях и прочих параметрах, выразив их в тензорном виде.

\section{Объект}
Представим висящий в безвоздушном пространстве объект в форме шара с близким к шаровому тензором инерции, на поверхности которого системно или же хаотично расположен набор реактивных или иной природы движителей. Двигатели создают силу или момент (или их пару) в направлении одной какой-либо оси. Оси двигателей в общем случае могут быть направлены в любых направлениях относительно центра масс шара. Подавая сигналы управления на двигатели шар может перемещаться в различных направлениях в пространстве, или же задавать себе вращательное движение, причем, если, двигатели расположены в достаточной степени разнообразно, то для выполнения любого желаемого набора главных векторов сил и момента всегда найдётся необходимая комбинация управляющих воздействий. В противном, вырожденном случае расположения движителей, шар может совершать только ограниченный набор эволюций.

Действие двигателей на положение объекта относительно неподвижного наблюдателя, очевидно, зависит от текущей ориентации объекта, то есть тот двигатель, что позволял минуту назад лететь вверх, после поворота объекта станет задавать силу, направленную вправо и так далее. Можно сказать, что в общем случае оси тяги и положения движителей могут изменяться даже в рамках собственной системы координат объекта. Предположим, например, что двигатели имеют ноги, бегают по поверхности шара и всячески крутят осями тяги. Даже в условиях такого непотребства, если двигателей достаточно, а объект имеет хороший вычислитель и знания о том, как расположены в текущий момент времени его двигатели и он сам в пространстве наблюдателя, он может найти и задать искомое управляющее воздействие. В случае вырожденного случая положения двигателей, управляющее воздействие должно соответствовать возможностям объекта управление.

Описанный сфереческий в вакууме конь, есть идеальная модель для множества разномастных систем и объектов управления, совершающих перемещения себя или части своих органов в пространстве. К числу таких объектов в первом приближении относятся роботы манипуляторы и дроны всех видов и расцветок (подводные, воздушные, космические). Все эти объекты характеризуются достаточно возможностью изменять положение и ориентацию в пространстве, а также большим набором неспециализированных исполнительных органов, влияние которых на положение зависит от текущей конфигурации объекта в пространстве.

\section{Идеальный дрон. Управление силами и моментами.}
Рассмотрим идеальный, абсолютно твёрдый объект с шаровым тензором инерции, имеющим безинерциальные органы управления, расположенные и организованные таким образом, что в совокупности, после приведения к центру масс объекта, могут создать пару главного вектора сил и главного вектора момента любого номинала и направления. Объект, разумеется, находится в безвоздушном пространстве, а все гравитационные силы скомпенсированы.

В рамках настоящей статьи мы не будем рассматривать уравнения движения объекта управления, поскольку удобный способ описания спинорной компоненты положения в контексте предлагаемой системы требует введения неклассической операции дифференцирования, связывающей тензор поворота с угловой скоростью объекта и введения дополнительного формализма, что несколько выходит за рамки настоящей статьи. Оставив эти выкладки для отдельной статьи, рассмотрим систему стабилизации положения объекта управления в трёхмерном пространстве.

Пусть объект находящийся в текущем $P$ абсолютном положении требуется переместить в положение уставки $U$.
Тензор положения $P$ объекта состоит из линейной $x_p$ и спинорной $\theta_p$ компонет. Расчитаем для текущего момента времени ошибку $E$, состоящую из линейной $x_e$ и спинорной $\theta_e$ ошибок позиционирования.

Уравнения динамики относительно ошибки позиционирования в системе координат наблюдателя в тензорном виде могут быть записаны следующим образом.
\begin{equation}
\frac{dv_p}{dt} = F_p  + \omega_o x v_p
\frac{dx_e}{dt} = v_u - v_p + \omega_o x x_e

\frac{dK_p}{dt} = M_p + \omega_o x K_p
\omega_p = I^-1 * K_p
\frac{d\theta_e}{dt} = Z(\theta)(\omega_u - \omega_p) + \omega_o x \theta_e
\end{equation}
В этих уравнениях $\omega_o$ - угловая скорость системы наблюдателя, $Z(\theta)$ - тензор Жилина. 

Для более простого случая, когда положение уставки и система наблюдателя неподвижны,   
\begin{equation}
\frac{dv_p}{dt} = F_p
\frac{dx_e}{dt} = - v_p

\frac{dK_p}{dt} = M_p
\omega_p = I^-1 * K_p
\frac{d\theta_e}{dt} = Z(\theta)(- \omega_p)
\end{equation}

Предположим, что мы умеем так расчитать управление на исполнительные органы, чтобы создать искомые векторы $F_p$ и $M_p$, что в условиях предположения их безинерциальности позволяет нам интерпретировать их, как сигналы управления. Замкнём контур управления через тензоры положения и угловой скорости.

(Учитывая, что $v_p = -v_e$ и $\theta_p = -\theta_e$)

 \begin{equation}
\frac{dv_e}{dt} = -K_1*x_e -K_2*v_e
\frac{dx_e}{dt} = v_e

\frac{d\theta_e}{dt} = Z(\theta)(\omega_e)
\frac{d\omega_e}{dt} = I^-1 * (-K_1*\theta_e -K_2*\omega_e)
\end{equation}

Если динамика компоненты линейного положения тривиальна, то динамика спинорной компоненты нелинейна, однако в предположениях, которые будут рассмотрены ниже, в рабочих режимах тензор жилина практически равен единичной матрице, а потому не вносит существенного вклада в динамику. Тензор инерции для шарообразного объекта также становится единичным, и его влияние также будет рассмотрено позже. Для нашего идеального случая динамика вращательного движения в первом приближении практически повторяет динамику линейного.

В целом можно сказать, что передаточная функция объекта управления по всем каналам имеет вид инерциального звена, замкнутого по параметрам положения и скорости. 

Рассмотрим передаточные функции СУ, линиаризованной по главным каналам - 3ём линейным и 3ём угловым, соответствующим главным осям объекта. 
Общий вид будет одинаков:

W_1 = 1/A*s^2
W_2 = K_2s + K_1

W_зам = W_1 * W_2 / W_1 * W_2 + 1

W_зам = K_1s + K_2 / A*s^2 + K_1s + K_2
W_зам = K_1/K_2s + 1 / A/K_2s^2 + K_1/K_2s + 1

Здесь A - инерционный параметр, то есть, или масса или один из главных моментов инерции. Мы видим, что линейные каналы идентичны, а спинорные различаются инерционным параметром. Таким образом мы имеем 4 главных канала системы управления.

% ЗАМКНУТАЯ СИСТЕМА!!!!
%$1/T_o*s**2$, что соответствует колебательному звену, а c учётом регулятора 
% $1/T_o*s**2 * \frac{T^2*s+T*\ksi*s}{1}$
 (Надо отметить, что T_o для линейного и спинорного каналов в общем случае отличается.)

Вычислительный эксперимент....

Точно также можно строить передаточные функции для управления объектами с иной динамикой. Например для объектов типа манипулятора, где мы управляем не силами но скоростями обобщенных координат манипулятора,
% $1*s * \frac{T*s+}{1}$


\section{Идеальный манипулятор. Скоростное управление.}
Рассмотрим идеальный, абсолютно твёрдый объект, имеющим безинерциальные органы управления, расположенные и организованные таким образом, что в совокупности могут создать пару линейной и угловой скорости любого номинала и направления.

Такой объект соответствует идеальному выходному звену манипулятора, приводимому в движение набором кинематических звеньев. Управление осуществляется параметрами скоростей, поскольку подчинённые системы управления сервоприводов кинематических пар решают задачу силового управления самостоятельно, требуя уставку в виде сигнала скорости (инкремента позиции). Если динамические возможности сервосистем достаточны, их передаточные функции можно считать единичными, что приближает нас к описанному идеальному объекту.

Поскольку в этом варианте мы управляем первой производной выходного сигнала, система упрощается.

\frac{dP_e}{dt} = V_u - V_p + \omega_o x P_e





\end{document}

