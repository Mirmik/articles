\documentclass[a4paper]{article}
\usepackage[12pt]{extsizes}
\usepackage[left=30mm, top=25mm, right=20mm, bottom=25mm, nohead]{geometry}
\usepackage{xcolor}
\usepackage{amssymb}
\usepackage{amsmath}
\usepackage{lipsum}
\setlength{\parindent}{3ex}
\setlength{\parskip}{1em}
\usepackage[utf8]{inputenc}
\usepackage[T2A]{fontenc}
\usepackage[english, russian]{babel}
\usepackage{caption}
\usepackage{graphicx} 
\graphicspath{ {./images/} }
\definecolor{shadecolor}{RGB}{230,230,230}
\begin{document}

%\section{Система управления идеальным изотропным объектом управления в трёхмерном пространстве.}

\section{Объект}
Представим висящий в безвоздушном пространстве объект в форме шара с близким к шаровому тензором инерции, на поверхности которого системно или же хаотично расположен набор реактивных или иной природы движителей. Двигатели создают силу вдоль одной какой-либо оси. Оси двигателей в общем случае могут быть направлены в любых направлениях относительно центра масс шара. Подавая сигналы управления на двигатели шар может перемещаться в различных направлениях в пространстве, или же задавать себе вращательное движение, причем, если, двигатели расположены в достаточной степени разнообразно, то для выполнения любого желаемого набора главных векторов сил и момента всегда найдётся необходимая комбинация управляющих воздействий.

Действие двигателей на положение объекта относительно неподвижного наблюдателя, очевидно, зависит от текущей ориентации объекта, то есть тот двигатель, что позволял минуту назад лететь вверх, после поворота объекта станет задавать силу, направленную вправо и так далее. Можно сказать, что в общем случае маршевые оси и положения движителей могут изменяться даже в рамках собственной системы координат объекта. Предположим, например, что двигатели имеют ноги, бегают по поверхности шара и всячески крутят маршевыми осями. Даже в условиях такого непотребства, если двигателей достаточно, а объект имеет хороший вычислитель и знания о том, как расположены в текущий момент времени его двигатели и он сам в пространстве наблюдателя, он может найти и задать искомое управляющее воздействие.

Данная работа, как вы, несомненно уже поняли, как раз и посвящена управлению таким любопытным объектом.

\section{Актуальность}
Естественный вопрос, который внимательный читатель, разумеется, хочет задать еще с первого абзаца: "А зачем управлять мифическим парнокопытным, имеющим форму, почетаемую совершенной в древней греции?".

Фокус в том, что сфереческий в вакууме конь, описанный ранее есть идеальная модель для множества разномастных систем и объектов управления, совершающих перемещения себя или части своих органов в пространстве. К числу таких объектов в первом приближении относятся роботы манипуляторы и дроны всех видов и расцветок (подводные, воздушные, космические). Все эти объекты характеризуются достаточно возможностью изменять положение и ориентацию в пространстве, а также большим набором неспециализированных исполнительных органов, влияние которых на положение зависит от текущей конфигурации.

На основе идеальной модели мы рассмотрим предлагаемый подход к решению, а также проблемы, возникающие при попытке построения системы управления этим семейством объектов.

\section{Первое приближение.}
Рассмотрим идеальный, абсолютно твёрдый объект с шаровым тензором инерции, имеющим безинерциальные органы управления, расположенные и организованные таким образом, что в совокупности, после приведения к центру масс объекта, могут создать пару главного вектора сил и главного вектора момента любого номинала и направления. Объект, разумеется, находится в безвоздушном пространстве, а все гравитационные силы скомпенсированы.

Пусть объект находящийся в текущем $P$ абсолютном положении требуется переместить в положение уставки $U$.
Тензор положения объекта состоит из линейной $X_p$ и спинорной $\theta_p$ компонет. Расчитаем для текущего момента времени ошибку $E$, состоящую из линейной $X_e$ и спинорной $\theta_e$ ошибок позиционирования.

Уравнения динамики относительно ошибки позиционирования в системе координат наблюдателя в тензорном виде могут быть записаны следующим образом.
\begin{equation}
\frac{dV_p}{dt} = F_p
\frac{dX_e}{dt} = V_u - V_p + \omega_o x X_e

\frac{dK_p}{dt} = M_p
\omega_p = I^-1 * K_p
\frac{d\theta_e}{dt} = Z(\theta)(\omega_u - \omega_p) + \omega_o x X_e
\end{equation}
В этих уравнениях $\omega_o$ - угловая скорость системы наблюдателя, $Z(\theta)$ - тензор Жилина. 

Для более простого случая, когда положение уставки и система наблюдателя неподвижны,   
\begin{equation}
\frac{dV_p}{dt} = F_p
\frac{dX_e}{dt} = - V_p

\frac{dK_p}{dt} = M_p
\omega_p = I^-1 * K_p
\frac{d\theta_e}{dt} = Z(\theta)(- \omega_p)
\end{equation}

Предположим, что мы умеем так расчитать управление на исполнительные органы, чтобы создать искомые векторы $F_p$ и $M_p$, что в условиях предположения их безинерциальности позволяет замкнуть через них контур управления. Замкнём контур управления через тензоры положения и угловой скорости.

 \begin{equation}
\frac{dX_e}{dt} = V_e
\frac{dV_e}{dt} = -K_1*X_e -K_2*V_e

\frac{d\theta_e}{dt} = Z(\theta)(\omega_e)
\frac{d\omega_p}{dt} = I^-1 * (-K_1*\theta_e -K_2*\omega_e)
\end{equation}

Если динамика компоненты линейного положения тривиальна, то динамика спинорной компоненты нелинейна, однако в
предположениях, которые будут рассмотрены ниже, в рабочих режимах тензор жилина практически равен единичной матрице, а потому не вносит существенного вклада в динамику. Тензор инерции для шарообразного объекта также становится единичным, и его влияние также будет рассмотрено позже. Для нашего идеального случая динамика вращательного движения в первом приближении практически повторяет динамику линейного.

В целом можно сказать, что передаточная функция объекта управления по всем каналам имеет вид 
% ЗАМКНУТАЯ СИСТЕМА!!!!
%$1/T_o*s**2$, что соответствует колебательному звену, а c учётом регулятора 
% $1/T_o*s**2 * \frac{T^2*s+T*\ksi*s}{1}$
 (Надо отметить, что T_o для линейного и спинорного каналов в общем случае отличается.)

Вычислительный эксперимент....

Точно также можно строить передаточные функции для управления объектами с иной динамикой. Например для объектов типа манипулятора, где мы управляем не силами но скоростями обобщенных координат манипулятора,
% $1*s * \frac{T*s+}{1}$









\end{document}

